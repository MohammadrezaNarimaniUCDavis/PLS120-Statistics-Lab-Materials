\documentclass[11pt,a4paper]{article}
\usepackage[utf8]{inputenc}
\usepackage[margin=1in]{geometry}
\usepackage{graphicx}
\usepackage[hidelinks]{hyperref}
\usepackage{xcolor}
\usepackage{fancyhdr}
\usepackage{titlesec}
\usepackage{enumitem}
\usepackage{tcolorbox}
\usepackage{fontawesome5}
\usepackage{amsmath}
\usepackage{amssymb}
\usepackage{lmodern}
\usepackage[T1]{fontenc}
\usepackage{needspace}

% Colors
\definecolor{primarygreen}{RGB}{46,125,50}
\definecolor{accentgreen}{RGB}{76,175,80}
\definecolor{lightgray}{RGB}{248,249,250}
\definecolor{bluestat}{RGB}{59,130,246}
\definecolor{redstat}{RGB}{239,68,68}
\definecolor{purplestat}{RGB}{147,51,234}
\definecolor{orangestat}{RGB}{249,115,22}

% Header and footer
\setlength{\headheight}{15pt}
\addtolength{\topmargin}{-3pt}
\pagestyle{fancy}
\fancyhf{}
\fancyhead[L]{\textcolor{primarygreen}{\textbf{PLS 120 - Week 9 Tutorial}}}
\fancyhead[R]{\textcolor{primarygreen}{UC Davis}}
\fancyfoot[C]{\thepage}

% Title formatting
\titleformat{\section}{\large\bfseries\color{primarygreen}}{}{0em}{}[\titlerule]
\titleformat{\subsection}{\normalsize\bfseries\color{primarygreen}}{}{0em}{}

% Custom boxes
\newtcolorbox{infobox}{
    colback=lightgray,
    colframe=primarygreen,
    boxrule=1pt,
    arc=3pt,
    left=10pt,
    right=10pt,
    top=10pt,
    bottom=10pt
}

\newtcolorbox{warningbox}{
    colback=accentgreen!10,
    colframe=primarygreen,
    boxrule=1pt,
    arc=3pt,
    left=10pt,
    right=10pt,
    top=10pt,
    bottom=10pt
}

\newtcolorbox{formulabox}{
    colback=bluestat!10,
    colframe=bluestat,
    boxrule=1pt,
    arc=3pt,
    left=10pt,
    right=10pt,
    top=10pt,
    bottom=10pt
}

\newtcolorbox{methodbox}{
    colback=purplestat!10,
    colframe=purplestat,
    boxrule=1pt,
    arc=3pt,
    left=10pt,
    right=10pt,
    top=10pt,
    bottom=10pt
}

\begin{document}

% Title page
\begin{titlepage}
    \centering
    \vspace*{2cm}
    
    {\Huge\bfseries\color{primarygreen} PLS 120: Applied Statistics in Agricultural Sciences}
    
    \vspace{1cm}
    
    {\Large\color{primarygreen} Tree Age Estimation Methods}
    
    \vspace{2cm}
    
    \includegraphics[width=0.3\textwidth]{../../images/logos/Home_Page_Logo.png}
    
    \vspace{2cm}
    
    {\large\bfseries Week 9 Tutorial Guide}
    
    \vspace{1cm}
    
    {\large Mohammadreza Narimani}\\
    {\normalsize Department of Biological and Agricultural Engineering}\\
    {\normalsize University of California, Davis}
    
    \vspace{1cm}
    
    {\normalsize mnarimani@ucdavis.edu}
    
    \vfill
    
    {\normalsize November 2024}
\end{titlepage}

{\small \tableofcontents}

\section{Important Links}

\begin{tcolorbox}[colback=accentgreen!20, colframe=primarygreen, boxrule=2pt, arc=5pt, title={\textbf{\Large Essential Course Resources}}]
\centering
\textbf{\Large Course Website}\\[0.5cm]
\textcolor{primarygreen}{\textbf{All course materials available at:}}\\[0.3cm]
\href{https://mohammadrezanarimaniucdavis.github.io/PLS120-Statistics-Lab-Materials/}{\textcolor{primarygreen}{\underline{Course Website Link}}}\\[0.8cm]

\textbf{\Large Interactive Binder Environment}\\[0.5cm]
\textcolor{primarygreen}{\textbf{Access Week 9 lab materials:}}\\[0.3cm]
\href{https://mybinder.org/v2/gh/MohammadrezaNarimaniUCDavis/PLS120-Statistics-Lab-Materials/binder-week9}{\textcolor{primarygreen}{\underline{Week 9 Binder Link}}}
\end{tcolorbox}

\section{Welcome to Week 9: Tree Age Estimation Methods}

This week focuses on \textbf{method comparison techniques} using real tree age data. You'll master confidence intervals, understand the difference between independent and paired t-tests, and learn to make statistical decisions in agricultural research contexts.

\section{Confidence Intervals}

\subsection{Understanding Confidence Intervals}

Confidence intervals provide a range of plausible values for a population parameter based on sample data.

\subsubsection{Confidence Interval Formula}

\begin{formulabox}
\textbf{95\% Confidence Interval for Mean:}
$$CI = \overline{x} \pm t_{\alpha/2, df} \times \frac{s}{\sqrt{n}}$$
Where:\\
\texttt{$\overline{x}$} = sample mean\\
\texttt{$t_{\alpha/2, df}$} = critical t-value\\
\texttt{$s$} = sample standard deviation\\
\texttt{$n$} = sample size\\
\texttt{$df = n - 1$} = degrees of freedom\\[0.3cm]
\textbf{R Implementation:}\\
\texttt{t.test(data)\$conf.int}\\
\texttt{mean(data)} $\pm$ \texttt{qt(0.975, df=n-1) * sd(data)/sqrt(n)}
\end{formulabox}

\section{Method Comparison}

\subsection{Independent vs Paired Comparisons}

Understanding when to use each approach is crucial for valid statistical inference.

\subsubsection{Independent Samples}

\begin{methodbox}
\textbf{Independent Two-Sample t-test:}\\
Use when comparing two separate groups\\
Example: Method A on trees 1-5, Method B on trees 6-10\\[0.3cm]
\textbf{Formula:}
$$t = \frac{\overline{x_1} - \overline{x_2}}{SE_{diff}}$$
Where: $SE_{diff} = \sqrt{\frac{s_1^2}{n_1} + \frac{s_2^2}{n_2}}$\\[0.3cm]
\textbf{R Implementation:}\\
\texttt{t.test(method\_A, method\_B, paired = FALSE)}
\end{methodbox}

\subsubsection{Paired Samples}

\begin{methodbox}
\textbf{Paired t-test:}\\
Use when same subjects measured with both methods\\
Example: Both methods applied to same 10 trees\\[0.3cm]
\textbf{Formula:}
$$t = \frac{\overline{d}}{s_d/\sqrt{n}}$$
Where: $\overline{d}$ = mean of differences, $s_d$ = SD of differences\\[0.3cm]
\textbf{R Implementation:}\\
\texttt{t.test(method\_A, method\_B, paired = TRUE)}\\
\texttt{differences <- method\_A - method\_B}\\
\texttt{t.test(differences)}
\end{methodbox}

\section{Assignment 9 Overview}

\subsection{Dataset: Tree Age Methods}

Compare two techniques for estimating tree age on 10 trees:
\begin{itemize}
    \item Method A: Traditional ring counting
    \item Method B: Modern imaging technique
    \item Same trees measured with both methods (paired design)
\end{itemize}

\subsection{Assignment Structure (20 points total)}

\begin{enumerate}
    \item \textbf{Question A: Means and Confidence Intervals (4 points)}
    \begin{itemize}
        \item Calculate means for both methods
        \item Compute 95\% confidence intervals
        \item Include manual calculations
    \end{itemize}
    
    \item \textbf{Question B: CI Interpretation (2 points)}
    \begin{itemize}
        \item Analyze confidence interval overlap
        \item Draw conclusions about method differences
    \end{itemize}
    
    \item \textbf{Question C: Difference Analysis (4 points)}
    \begin{itemize}
        \item Calculate mean difference between methods
        \item Determine 95\% CI for the difference
        \item Use pooled variance approach
    \end{itemize}
    
    \item \textbf{Question D: Significance Assessment (2 points)}
    \begin{itemize}
        \item Interpret CI for difference
        \item Determine statistical significance
    \end{itemize}
    
    \item \textbf{Question E: Independent t-test (4 points)}
    \begin{itemize}
        \item Perform two-sample t-test
        \item Manual t-statistic calculation
        \item Hypothesis testing decision
    \end{itemize}
    
    \item \textbf{Question F: Paired t-test (4 points)}
    \begin{itemize}
        \item Perform paired samples analysis
        \item Calculate differences array
        \item Compare with independent test results
    \end{itemize}
\end{enumerate}

\section{Key Statistical Concepts}

\subsection{Confidence Interval Interpretation}

\begin{formulabox}
\textbf{Interpreting Confidence Intervals:}\\[0.3cm]
\textbf{Individual CIs:}\\
We are 95\% confident the true mean lies within the interval\\[0.3cm]
\textbf{Overlapping CIs:}\\
Suggests no significant difference between methods\\
However, overlap doesn't guarantee non-significance\\[0.3cm]
\textbf{CI for Difference:}\\
If CI includes 0 → No significant difference\\
If CI excludes 0 → Significant difference\\[0.3cm]
\textbf{Example Interpretation:}\\
Method A: 95\% CI (22.3, 26.5)\\
Method B: 95\% CI (19.9, 24.1)\\
Difference: 95\% CI (-0.4, 5.2) includes 0 → Not significant
\end{formulabox}

\subsection{Test Selection Guide}

\begin{warningbox}
\textbf{When to Use Each Test:}
\begin{itemize}
    \item \textbf{Independent t-test:} Different subjects in each group
    \item \textbf{Paired t-test:} Same subjects measured twice
    \item \textbf{One-sample t-test:} Compare sample to known value
\end{itemize}
\textbf{Tree Age Example:}\\
Since both methods were used on the same trees, paired t-test is appropriate. However, we also demonstrate independent t-test for comparison.
\end{warningbox}

\section{Manual Calculations}

\subsection{Step-by-Step Calculations}

\subsubsection{Confidence Interval Calculation}

\begin{formulabox}
\textbf{Example: Method A Confidence Interval}\\[0.3cm]
Given: $n = 10$, $\overline{x} = 24.4$, $s = 2.95$\\[0.3cm]
\textbf{Step 1:} Calculate standard error\\
$SE = \frac{s}{\sqrt{n}} = \frac{2.95}{\sqrt{10}} = 0.93$\\[0.3cm]
\textbf{Step 2:} Find t-critical value\\
$t_{0.025, 9} = 2.262$ (from t-table)\\[0.3cm]
\textbf{Step 3:} Calculate margin of error\\
$ME = t \times SE = 2.262 \times 0.93 = 2.10$\\[0.3cm]
\textbf{Step 4:} Construct interval\\
$CI = 24.4 \pm 2.10 = (22.30, 26.50)$
\end{formulabox}

\section{Special Features This Week}

\subsection{Complete Solutions Provided}

\begin{infobox}
\textbf{Learning by Running Code:}
\begin{itemize}
    \item \textbf{All code cells} contain complete working solutions
    \item \textbf{Manual calculations} alongside R functions
    \item \textbf{Step-by-step explanations} in comments
    \item \textbf{Decision logic} with if/else statements
    \item \textbf{Formatted output} for easy interpretation
\end{itemize}
\textbf{Educational Benefits:}
\begin{itemize}
    \item See immediate results by running cells
    \item Compare manual vs automated calculations
    \item Understand statistical decision-making process
    \item Learn proper result interpretation
\end{itemize}
\end{infobox}

\section{Getting Started}

\begin{enumerate}
    \item Launch Week 9 Binder environment
    \item Navigate to \texttt{assignment} folder
    \item Open \texttt{Assignment9.ipynb}
    \item Run each code cell to see complete solutions
    \item Learn by observing calculations and interpretations
\end{enumerate}

\section{Learning Objectives}

By the end of this week, you will be able to:
\begin{itemize}
    \item Calculate and interpret confidence intervals for means
    \item Compare two measurement methods statistically
    \item Choose between independent and paired t-tests appropriately
    \item Perform manual statistical calculations
    \item Interpret confidence intervals for differences
    \item Make statistical decisions based on evidence
    \item Understand the relationship between CIs and hypothesis tests
    \item Apply method comparison techniques to agricultural research
\end{itemize}

\section{Tips for Success}

\begin{warningbox}
\textbf{Best Practices:}
\begin{itemize}
    \item Run each code cell sequentially to see the complete analysis
    \item Pay attention to manual calculations vs R function results
    \item Understand when confidence intervals suggest significance
    \item Note the difference between independent and paired test results
    \item Focus on proper interpretation of statistical output
    \item Consider practical significance alongside statistical significance
\end{itemize}
\end{warningbox}

\section{Need Help?}

\begin{infobox}
\textbf{Mohammadreza Narimani}\\
Email: mnarimani@ucdavis.edu\\
Department of Biological and Agricultural Engineering, UC Davis\\
Office Hours: Thursdays 10 AM - 12 PM (Zoom)\\
Zoom Link: \href{https://ucdavis.zoom.us/j/99533096447}{Join Office Hours}
\end{infobox}

\vfill

\begin{center}
\textit{Last updated: November 2024 | PLS 120 - Applied Statistics in Agriculture | UC Davis}\\
\textit{Week 9: Tree Age Estimation Methods}
\end{center}

\end{document}