\documentclass[11pt,a4paper]{article}
\usepackage[utf8]{inputenc}
\usepackage[margin=1in]{geometry}
\usepackage{graphicx}
\usepackage[hidelinks]{hyperref}
\usepackage{xcolor}
\usepackage{fancyhdr}
\usepackage{titlesec}
\usepackage{enumitem}
\usepackage{tcolorbox}
\usepackage{fontawesome5}
\usepackage{amsmath}
\usepackage{amssymb}
\usepackage{lmodern}
\usepackage[T1]{fontenc}
\usepackage{needspace}

% Colors
\definecolor{primarygreen}{RGB}{46,125,50}
\definecolor{accentgreen}{RGB}{76,175,80}
\definecolor{lightgray}{RGB}{248,249,250}
\definecolor{bluestat}{RGB}{59,130,246}
\definecolor{redstat}{RGB}{239,68,68}
\definecolor{purplestat}{RGB}{147,51,234}

% Header and footer
\setlength{\headheight}{15pt}
\addtolength{\topmargin}{-3pt}
\pagestyle{fancy}
\fancyhf{}
\fancyhead[L]{\textcolor{primarygreen}{\textbf{PLS 120 - Week 5 Tutorial}}}
\fancyhead[R]{\textcolor{primarygreen}{UC Davis}}
\fancyfoot[C]{\thepage}

% Title formatting
\titleformat{\section}{\large\bfseries\color{primarygreen}}{}{0em}{}[\titlerule]
\titleformat{\subsection}{\normalsize\bfseries\color{primarygreen}}{}{0em}{}

% Custom boxes
\newtcolorbox{infobox}{
    colback=lightgray,
    colframe=primarygreen,
    boxrule=1pt,
    arc=3pt,
    left=10pt,
    right=10pt,
    top=10pt,
    bottom=10pt
}

\newtcolorbox{warningbox}{
    colback=accentgreen!10,
    colframe=primarygreen,
    boxrule=1pt,
    arc=3pt,
    left=10pt,
    right=10pt,
    top=10pt,
    bottom=10pt
}

\newtcolorbox{formulabox}{
    colback=bluestat!10,
    colframe=bluestat,
    boxrule=1pt,
    arc=3pt,
    left=10pt,
    right=10pt,
    top=10pt,
    bottom=10pt
}

\newtcolorbox{estimationbox}{
    colback=purplestat!10,
    colframe=purplestat,
    boxrule=1pt,
    arc=3pt,
    left=10pt,
    right=10pt,
    top=10pt,
    bottom=10pt
}

\begin{document}

% Title page
\begin{titlepage}
    \centering
    \vspace*{2cm}
    
    {\Huge\bfseries\color{primarygreen} PLS 120: Applied Statistics in Agricultural Sciences}
    
    \vspace{1cm}
    
    {\Large\color{primarygreen} Sampling and Estimation}
    
    \vspace{2cm}
    
    \includegraphics[width=0.3\textwidth]{../../images/logos/Home_Page_Logo.png}
    
    \vspace{2cm}
    
    {\large\bfseries Week 5 Tutorial Guide}
    
    \vspace{1cm}
    
    {\large Mohammadreza Narimani}\\
    {\normalsize Department of Biological and Agricultural Engineering}\\
    {\normalsize University of California, Davis}
    
    \vspace{1cm}
    
    {\normalsize mnarimani@ucdavis.edu}
    
    \vfill
    
    {\normalsize October 2025}
\end{titlepage}

{\small \tableofcontents}

\section{Important Links}

\begin{tcolorbox}[colback=accentgreen!20, colframe=primarygreen, boxrule=2pt, arc=5pt, title={\textbf{\Large Essential Course Resources}}]
\centering
\textbf{\Large Course Website}\\[0.5cm]
\textcolor{primarygreen}{\textbf{All course materials available at:}}\\[0.3cm]
\href{https://mohammadrezanarimaniucdavis.github.io/PLS120-Statistics-Lab-Materials/}{\textcolor{primarygreen}{\underline{Course Website Link}}}\\[0.8cm]

\textbf{\Large Interactive Binder Environment}\\[0.5cm]
\textcolor{primarygreen}{\textbf{Access Week 5 lab materials:}}\\[0.3cm]
\href{https://mybinder.org/v2/gh/MohammadrezaNarimaniUCDavis/PLS120-Statistics-Lab-Materials/binder-week5}{\textcolor{primarygreen}{\underline{Week 5 Binder Link}}}
\end{tcolorbox}

\section{Welcome to Week 5: Sampling and Estimation}

This week, we explore \textbf{sampling distributions and statistical estimation} - critical skills for designing agricultural studies and interpreting research results. You'll learn to calculate sample sizes, construct confidence intervals, and understand estimation precision!

\section{Reproducible Research with set.seed()}

\subsection{Understanding Reproducible Research}

The set.seed() function is essential for creating reproducible research, especially when involving random number generation and sampling.

\subsubsection{Why Reproducibility Matters}

\begin{formulabox}
\textbf{Benefits of Reproducible Research:}\\
• Allows others to verify your results\\
• Enables debugging of statistical analyses\\
• Ensures consistent results across runs\\
• Required for scientific publication\\[0.3cm]
\textbf{Usage:} \texttt{set.seed(value)}\\
\textbf{Example:} \texttt{set.seed(123)}\\
The specific number doesn't matter - use memorable values like 123, 42, or 2025
\end{formulabox}

\needspace{5\baselineskip}
\section{Z-Score Standardization}

\subsection{Understanding Z-Scores}

Z-scores help us \textbf{standardize} data by converting values to standard deviations from the mean, making it easier to compare different datasets and calculate probabilities.

\subsubsection{Z-Score Formula}

\begin{formulabox}
\textbf{Z-Score Formula:}
$$z = \frac{x - \mu}{\sigma}$$
Where:\\
\texttt{x} = individual value\\
\texttt{$\mu$} = population mean\\
\texttt{$\sigma$} = population standard deviation\\[0.3cm]
\textbf{R Implementation:}\\
\texttt{z\_scores <- (data - mean(data)) / sd(data)}\\
\texttt{z\_scores <- scale(data)} \# Alternative method
\end{formulabox}

\subsection{Normal Distribution Functions}

\subsubsection{Essential Normal Functions}

\begin{formulabox}
\textbf{pnorm() - Cumulative Probability:}\\
\texttt{pnorm(q, mean = 0, sd = 1)}\\
Returns: P(X $\leq$ q) for normal distribution\\[0.3cm]
\textbf{qnorm() - Quantiles:}\\
\texttt{qnorm(p, mean = 0, sd = 1)}\\
Returns: Value x such that P(X $\leq$ x) = p\\[0.3cm]
\textbf{rnorm() - Random Generation:}\\
\texttt{rnorm(n, mean = 0, sd = 1)}\\
Generates n random values from normal distribution
\end{formulabox}

\needspace{5\baselineskip}
\section{Confidence Intervals}

\subsection{Understanding Confidence Intervals}

Confidence intervals provide an estimated range that is likely to include the true value of an unknown population parameter with a certain level of confidence.

\subsubsection{Confidence Interval Formula}

\begin{estimationbox}
\textbf{Confidence Interval for Means:}
$$CI = \overline{x} \pm z_{\alpha/2} \times \frac{s}{\sqrt{n}}$$
Where:\\
\texttt{$\overline{x}$} = sample mean\\
\texttt{z} = critical z-value\\
\texttt{s} = sample standard deviation\\
\texttt{n} = sample size\\[0.3cm]
\textbf{Common Z-Values:}\\
90\% confidence: z = 1.645\\
95\% confidence: z = 1.96\\
99\% confidence: z = 2.576
\end{estimationbox}

\subsubsection{Margin of Error}

\begin{estimationbox}
\textbf{Margin of Error Formula:}
$$ME = z_{\alpha/2} \times \frac{s}{\sqrt{n}}$$
\textbf{R Implementation:}\\
\texttt{alpha <- 0.05}\\
\texttt{z\_score <- qnorm(1 - alpha / 2)}\\
\texttt{margin\_error <- z\_score * (sample\_sd / sqrt(sample\_size))}\\
\texttt{lower\_bound <- sample\_mean - margin\_error}\\
\texttt{upper\_bound <- sample\_mean + margin\_error}
\end{estimationbox}

\needspace{5\baselineskip}
\section{Sample Size Calculations}

\subsection{Determining Required Sample Size}

When designing experiments, calculating appropriate sample size is critical to minimize error and ensure sufficient statistical power.

\subsubsection{Sample Size Formula for Proportions}

\begin{formulabox}
\textbf{Sample Size Formula:}
$$n = \frac{z^2 \times p \times (1-p)}{d^2}$$
Where:\\
\texttt{n} = required sample size\\
\texttt{z} = z-score for desired confidence level\\
\texttt{p} = expected proportion\\
\texttt{d} = desired margin of error\\[0.3cm]
\textbf{R Implementation:}\\
\texttt{z\_score <- qnorm(1 - alpha / 2)}\\
\texttt{sample\_size <- z\_score\^{}2 * p * (1 - p) / d\^{}2}\\
\texttt{ceiling(sample\_size)} \# Round up to whole number
\end{formulabox}

\subsection{Factors Affecting Sample Size}

\subsubsection{Trade-offs in Sample Size Planning}

\begin{estimationbox}
\textbf{Key Relationships:}\\
• Higher confidence level → Larger sample size needed\\
• Lower margin of error → Larger sample size needed\\
• Higher expected proportion (up to 0.5) → Larger sample size needed\\
• Cost vs. Precision: Balance practical constraints with statistical requirements\\[0.3cm]
\textbf{Example Scenarios:}\\
90\% confidence, 5\% margin: n $\approx$ 271\\
95\% confidence, 5\% margin: n $\approx$ 385\\
99\% confidence, 1\% margin: n $\approx$ 16,641
\end{estimationbox}

\needspace{5\baselineskip}
\section{Assignment 5 Overview}

\subsection{Assignment Structure (20 points total)}

\begin{enumerate}
    \item \textbf{Question 1: Basic Sample Size Calculation (6 points)}
    \begin{itemize}
        \item Calculate sample size for 95\% confidence and 5\% margin of error
        \item Explain margin of error concept
        \item Interpret confidence intervals correctly
    \end{itemize}
    
    \item \textbf{Question 2: Effect of Prevalence Rate (5 points)}
    \begin{itemize}
        \item Calculate sample sizes for different prevalence rates (0.5X, 2X, 3X)
        \item Analyze how prevalence affects sample size requirements
    \end{itemize}
    
    \item \textbf{Question 3: Effect of Margin of Error (5 points)}
    \begin{itemize}
        \item Calculate sample sizes for 2.5\% and 7.5\% margins of error
        \item Understand precision trade-offs in sample size planning
    \end{itemize}
    
    \item \textbf{Question 4: Effect of Confidence Level (4 points)}
    \begin{itemize}
        \item Compare sample sizes for 90\% and 99\% confidence levels
        \item Analyze how confidence level affects sample size
    \end{itemize}
\end{enumerate}

\needspace{5\baselineskip}
\section{Agricultural Applications}

\begin{infobox}
\textbf{Real-World Applications:}
\begin{itemize}
    \item \textbf{Seed Germination Studies} - Estimate germination rates with confidence intervals for planting density planning
    \item \textbf{Crop Yield Estimation} - Sample fields to predict total harvest with known precision
    \item \textbf{Pest Occurrence Surveys} - Calculate sample sizes needed to detect pest outbreaks reliably
    \item \textbf{Quality Control Testing} - Design sampling plans to ensure product quality standards
    \item \textbf{Weather Pattern Analysis} - Use historical data to create prediction intervals for rainfall
    \item \textbf{Agricultural Insurance} - Quantify risks using sampling and probability distributions
\end{itemize}
\end{infobox}

\section{Key Concepts Summary}

\subsection{Estimation Fundamentals}

\begin{estimationbox}
\textbf{Confidence Interval Interpretation:}\\
• A 95\% confidence interval means: if we repeated our study 100 times, about 95 of the intervals would contain the true population parameter\\
• Your specific interval either contains $\mu$ or it doesn't - the confidence is about the method\\
• Wider intervals = more confidence, narrower intervals = more precision\\[0.3cm]
\textbf{Sample Size Principles:}\\
• Larger samples create more precise estimates\\
• There's always a trade-off between precision and cost\\
• Plan sample size before data collection, not after
\end{estimationbox}

\subsection{Statistical Inference}

\begin{estimationbox}
\textbf{Population vs Sample:}\\
Population = entire group of interest\\
Sample = subset used to make inferences about population\\[0.3cm]
\textbf{Point Estimate vs Interval Estimate:}\\
Point estimate = single value (e.g., sample mean)\\
Interval estimate = range of plausible values (confidence interval)\\[0.3cm]
\textbf{Sampling Distribution:}\\
Distribution of sample statistics across many samples from the same population
\end{estimationbox}

\section{Getting Started}

\begin{enumerate}
    \item Launch Week 5 Binder environment
    \item Navigate to \texttt{class\_activity} folder
    \item Open \texttt{Week5\_Sampling\_Estimation.ipynb}
    \item Work through interactive exercises
    \item Complete Assignment 5 in \texttt{assignment} folder
\end{enumerate}

\section{Learning Objectives}

By the end of this week, you will be able to:
\begin{itemize}
    \item Calculate required sample sizes for different precision levels
    \item Construct confidence intervals for means and proportions
    \item Understand the relationship between confidence level and margin of error
    \item Use normal distribution functions for statistical inference
    \item Apply z-score standardization to compare different datasets
    \item Design studies with appropriate statistical power
    \item Make informed decisions about resource allocation in research
\end{itemize}

\section{Tips for Success}

\begin{warningbox}
\textbf{Best Practices:}
\begin{itemize}
    \item Use \texttt{ceiling()} to round up sample sizes to whole numbers
    \item Remember key z-values: 1.96 for 95\%, 1.645 for 90\%, 2.576 for 99\%
    \item Understand trade-offs: higher confidence = larger sample size
    \item Verify calculations by checking formulas and parameter values
    \item Think practically about cost and feasibility of sample sizes
    \item Always interpret confidence intervals in context
\end{itemize}
\end{warningbox}

\section{Need Help?}

\begin{infobox}
\textbf{Mohammadreza Narimani}\\
Email: mnarimani@ucdavis.edu\\
Department of Biological and Agricultural Engineering, UC Davis\\
Office Hours: Thursdays 10 AM - 12 PM (Zoom)\\
Zoom Link: \href{https://ucdavis.zoom.us/j/99533096447}{Join Office Hours}
\end{infobox}

\vfill

\begin{center}
\textit{Last updated: October 2025 | PLS 120 - Applied Statistics in Agriculture | UC Davis}\\
\textit{Week 5: Sampling and Estimation}
\end{center}

\end{document}