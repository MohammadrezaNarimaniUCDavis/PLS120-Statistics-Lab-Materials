\documentclass[11pt,a4paper]{article}
\usepackage[utf8]{inputenc}
\usepackage[margin=1in]{geometry}
\usepackage{graphicx}
\usepackage[hidelinks]{hyperref}
\usepackage{xcolor}
\usepackage{fancyhdr}
\usepackage{titlesec}
\usepackage{enumitem}
\usepackage{tcolorbox}
\usepackage{fontawesome5}
\usepackage{amsmath}
\usepackage{amssymb}
\usepackage{lmodern}
\usepackage[T1]{fontenc}
\usepackage{needspace}

% Colors
\definecolor{primarygreen}{RGB}{46,125,50}
\definecolor{accentgreen}{RGB}{76,175,80}
\definecolor{lightgray}{RGB}{248,249,250}
\definecolor{bluestat}{RGB}{59,130,246}
\definecolor{redstat}{RGB}{239,68,68}
\definecolor{purplestat}{RGB}{147,51,234}
\definecolor{goldproject}{RGB}{255,193,7}

% Header and footer
\setlength{\headheight}{15pt}
\addtolength{\topmargin}{-3pt}
\pagestyle{fancy}
\fancyhf{}
\fancyhead[L]{\textcolor{primarygreen}{\textbf{PLS 120 - Final Project Guide}}}
\fancyhead[R]{\textcolor{primarygreen}{UC Davis}}
\fancyfoot[C]{\thepage}

% Title formatting
\titleformat{\section}{\large\bfseries\color{primarygreen}}{}{0em}{}[\titlerule]
\titleformat{\subsection}{\normalsize\bfseries\color{primarygreen}}{}{0em}{}

% Custom boxes
\newtcolorbox{infobox}{
    colback=lightgray,
    colframe=primarygreen,
    boxrule=1pt,
    arc=3pt,
    left=10pt,
    right=10pt,
    top=10pt,
    bottom=10pt
}

\newtcolorbox{warningbox}{
    colback=accentgreen!10,
    colframe=primarygreen,
    boxrule=1pt,
    arc=3pt,
    left=10pt,
    right=10pt,
    top=10pt,
    bottom=10pt
}

\newtcolorbox{projectbox}{
    colback=goldproject!10,
    colframe=goldproject!80,
    boxrule=2pt,
    arc=3pt,
    left=10pt,
    right=10pt,
    top=10pt,
    bottom=10pt
}

\newtcolorbox{analysisbox}{
    colback=bluestat!10,
    colframe=bluestat,
    boxrule=1pt,
    arc=3pt,
    left=10pt,
    right=10pt,
    top=10pt,
    bottom=10pt
}

\newtcolorbox{submissionbox}{
    colback=purplestat!10,
    colframe=purplestat,
    boxrule=2pt,
    arc=3pt,
    left=10pt,
    right=10pt,
    top=10pt,
    bottom=10pt
}

\begin{document}

% Title page
\begin{titlepage}
    \centering
    \vspace*{2cm}
    
    {\Huge\bfseries\color{primarygreen} PLS 120: Applied Statistics in Agricultural Sciences}
    
    \vspace{1cm}
    
    {\Large\color{goldproject!80} Final Project Workspace}
    
    \vspace{0.5cm}
    
    {\normalsize\color{redstat} \textbf{Optional Alternative to Final Exam}}
    
    \vspace{2cm}
    
    \includegraphics[width=0.3\textwidth]{../../images/logos/Home_Page_Logo.png}
    
    \vspace{2cm}
    
    {\large\bfseries Complete Statistical Analysis Guide}
    
    \vspace{1cm}
    
    {\large Mohammadreza Narimani}\\
    {\normalsize Department of Biological and Agricultural Engineering}\\
    {\normalsize University of California, Davis}
    
    \vspace{1cm}
    
    {\normalsize mnarimani@ucdavis.edu}
    
    \vfill
    
    {\normalsize November 2024}
\end{titlepage}

{\small \tableofcontents}

\section{Important Links}

\begin{tcolorbox}[colback=goldproject!20, colframe=primarygreen, boxrule=2pt, arc=5pt, title={\textbf{\Large Essential Project Resources}}]
\centering
\textbf{\Large Course Website}\\[0.5cm]
\textcolor{primarygreen}{\textbf{All course materials available at:}}\\[0.3cm]
\href{https://mohammadrezanarimaniucdavis.github.io/PLS120-Statistics-Lab-Materials/}{\textcolor{primarygreen}{\underline{Course Website Link}}}\\[0.8cm]

\textbf{\Large Final Project Binder Environment}\\[0.5cm]
\textcolor{primarygreen}{\textbf{Access your project workspace:}}\\[0.3cm]
\href{https://mybinder.org/v2/gh/MohammadrezaNarimaniUCDavis/PLS120-Statistics-Lab-Materials/binder-final-project}{\textcolor{primarygreen}{\underline{Final Project Binder Link}}}
\end{tcolorbox}

\section{Welcome to Your Final Project}

This comprehensive final project is an \textbf{optional alternative to the final exam}. Students may choose to complete either the traditional final exam or this hands-on project. The project brings together all statistical concepts learned throughout PLS 120, allowing you to conduct a complete statistical analysis using your own agricultural dataset while demonstrating mastery of R programming, statistical testing, and scientific communication.

\section{Project Overview}

\subsection{Project Option}

\begin{projectbox}
\textbf{Choose Your Assessment Method:}\\
\textcolor{redstat}{\textbf{Option 1:}} Traditional Final Exam (150 points)\\
\textcolor{redstat}{\textbf{Option 2:}} Final Project Workspace (150 points)\\[0.5cm]
\textbf{This project is completely optional!} Students who prefer hands-on analysis over traditional testing may choose this comprehensive project as their final assessment.
\end{projectbox}

\subsection{Project Objectives}

\begin{projectbox}
\textbf{Demonstrate Mastery Of:}
\begin{itemize}
    \item \textbf{Data Management} - Import, clean, and organize agricultural datasets
    \item \textbf{Exploratory Analysis} - Summarize and visualize data patterns
    \item \textbf{Statistical Testing} - Apply appropriate tests for research questions
    \item \textbf{Interpretation} - Draw meaningful conclusions from results
    \item \textbf{Communication} - Present findings in professional format
    \item \textbf{R Programming} - Use functions, packages, and best practices
\end{itemize}
\end{projectbox}

\subsection{Pre-Installed Analysis Tools}

Your Binder environment includes comprehensive R packages for professional statistical analysis:

\begin{analysisbox}
\textbf{Data Manipulation \& Import:}\\
\texttt{dplyr, tidyr, readr, readxl, data.table}\\[0.3cm]
\textbf{Statistical Analysis:}\\
\texttt{car, agricolae, multcomp, emmeans, broom, lmtest, nortest}\\[0.3cm]
\textbf{Visualization:}\\
\texttt{ggplot2, plotly, corrplot, pheatmap}\\[0.3cm]
\textbf{Report Generation:}\\
\texttt{knitr, rmarkdown, DT}\\[0.3cm]
\textbf{Advanced Modeling:}\\
\texttt{MASS, lawstat} (plus ability to install additional packages)
\end{analysisbox}

\needspace{5\baselineskip}
\section{Getting Started}

\subsection{Step 1: Launch Your Environment}

\begin{enumerate}
    \item Click the \textbf{Final Project Binder Link} above
    \item Wait 2-5 minutes for environment setup
    \item Navigate to \texttt{Final\_Project\_Workspace.ipynb}
    \item Begin your analysis following the structured workflow
\end{enumerate}

\subsection{Step 2: Upload Your Data}

\begin{projectbox}
\textbf{Data Upload Process:}
\begin{enumerate}
    \item Use the \textbf{Upload button} in left panel (folder icon area)
    \item Select your data file (CSV, Excel, or other formats)
    \item File appears in left panel when ready
    \item Load data using appropriate R functions:
    \begin{itemize}
        \item CSV: \texttt{data <- read.csv("filename.csv")}
        \item Excel: \texttt{data <- read\_excel("filename.xlsx")}
    \end{itemize}
\end{enumerate}
\textbf{Supported Data Formats:}\\
CSV, Excel (.xlsx, .xls), Tab-delimited, and other common formats
\end{projectbox}

\needspace{5\baselineskip}
\section{Project Workflow}

\subsection{Phase 1: Data Preparation \& Exploration}

\begin{analysisbox}
\textbf{Essential Steps:}
\begin{enumerate}
    \item \textbf{Load Required Libraries}
    \begin{itemize}
        \item Load \texttt{ggplot2, dplyr, readr, knitr}
        \item Add specialized packages as needed
    \end{itemize}
    
    \item \textbf{Import and Examine Data}
    \begin{itemize}
        \item Use \texttt{str(data)} to check structure
        \item Apply \texttt{head(data)} and \texttt{summary(data)}
        \item Check for missing values with \texttt{is.na()}
    \end{itemize}
    
    \item \textbf{Data Cleaning}
    \begin{itemize}
        \item Convert categorical variables to factors
        \item Handle missing values appropriately
        \item Check for and address outliers
    \end{itemize}
    
    \item \textbf{Exploratory Visualization}
    \begin{itemize}
        \item Create histograms for continuous variables
        \item Generate boxplots for group comparisons
        \item Build scatter plots for relationships
    \end{itemize}
\end{enumerate}
\end{analysisbox}

\subsection{Phase 2: Descriptive Statistics}

\begin{analysisbox}
\textbf{Calculate Summary Statistics:}
\begin{itemize}
    \item \textbf{Central Tendency:} mean, median, mode
    \item \textbf{Variability:} variance, standard deviation, coefficient of variation
    \item \textbf{Distribution:} quantiles, skewness, normality tests
    \item \textbf{Group Comparisons:} statistics by treatment/category
\end{itemize}
\textbf{Key R Functions:}\\
\texttt{mean(), median(), var(), sd(), quantile(), summary()}\\
\texttt{group\_by() \%>\% summarise()} for grouped statistics
\end{analysisbox}

\subsection{Phase 3: Statistical Testing}

\begin{analysisbox}
\textbf{Choose Appropriate Tests Based on Data:}
\begin{itemize}
    \item \textbf{Two-sample t-tests:} Compare means between groups
    \item \textbf{ANOVA:} Compare means across multiple groups
    \item \textbf{Chi-square tests:} Test associations in categorical data
    \item \textbf{Correlation analysis:} Examine relationships between variables
    \item \textbf{Regression analysis:} Model relationships and predictions
\end{itemize}
\textbf{Hypothesis Testing Framework:}
\begin{enumerate}
    \item State null and alternative hypotheses
    \item Check test assumptions
    \item Perform appropriate statistical test
    \item Interpret p-values and confidence intervals
    \item Draw conclusions in agricultural context
\end{enumerate}
\end{analysisbox}

\subsection{Phase 4: Advanced Analysis}

\begin{analysisbox}
\textbf{Sophisticated Statistical Methods:}
\begin{itemize}
    \item \textbf{Multiple Regression:} Model complex relationships
    \item \textbf{ANOVA with Post-hoc Tests:} Detailed group comparisons
    \item \textbf{Non-parametric Tests:} When assumptions aren't met
    \item \textbf{Effect Size Calculations:} Practical significance assessment
    \item \textbf{Power Analysis:} Sample size and study design evaluation
\end{itemize}
\textbf{Model Validation:}\\
Check residuals, assess assumptions, validate predictions
\end{analysisbox}

\subsection{Phase 5: Results Visualization}

\begin{analysisbox}
\textbf{Publication-Quality Graphics:}
\begin{itemize}
    \item \textbf{Treatment Comparisons:} Boxplots with significance indicators
    \item \textbf{Relationships:} Scatter plots with regression lines
    \item \textbf{Distributions:} Histograms and density plots
    \item \textbf{Correlations:} Correlation matrices and heatmaps
    \item \textbf{Model Results:} Coefficient plots and diagnostic plots
\end{itemize}
\textbf{Visualization Best Practices:}\\
Clear titles, axis labels, legends, and appropriate color schemes
\end{analysisbox}

\needspace{5\baselineskip}
\section{Project Deliverables}

\subsection{Required Components}

\begin{submissionbox}
\textbf{Your Final Project Must Include:}
\begin{enumerate}
    \item \textbf{Student Information} (Complete in raw text cells)
    \item \textbf{Data Description} (Source, variables, sample size)
    \item \textbf{Research Questions} (Clear, testable hypotheses)
    \item \textbf{Statistical Analysis} (Appropriate tests with justification)
    \item \textbf{Results Visualization} (Professional plots and tables)
    \item \textbf{Interpretation} (Agricultural significance and implications)
    \item \textbf{Limitations} (Study constraints and future directions)
    \item \textbf{Conclusions} (Evidence-based recommendations)
\end{enumerate}
\end{submissionbox}

\subsection{Submission Requirements}

\begin{submissionbox}
\textbf{Submit TWO Files to Canvas:}
\begin{enumerate}
    \item \textbf{HTML Report} (File → Save and Export Notebook As → HTML)
    \begin{itemize}
        \item Complete formatted report with all outputs
        \item Professional presentation of results
        \item All plots and tables included
    \end{itemize}
    
    \item \textbf{Jupyter Notebook} (File → Download → .ipynb)
    \begin{itemize}
        \item Complete code for reproducibility
        \item Backup of your analysis
        \item All cells executed with outputs
    \end{itemize}
\end{enumerate}
\textbf{File Naming Convention:}\\
\texttt{LastName\_FirstName\_PLS120\_FinalProject.html}\\
\texttt{LastName\_FirstName\_PLS120\_FinalProject.ipynb}
\end{submissionbox}

\needspace{5\baselineskip}
\section{Evaluation Criteria}

\subsection{Grading Rubric}

\begin{projectbox}
\textbf{Technical Competency (40\%):}
\begin{itemize}
    \item Appropriate statistical test selection
    \item Correct R code implementation
    \item Proper assumption checking
    \item Accurate calculations and outputs
\end{itemize}

\textbf{Data Analysis Quality (30\%):}
\begin{itemize}
    \item Thorough exploratory analysis
    \item Meaningful visualizations
    \item Comprehensive statistical testing
    \item Appropriate model validation
\end{itemize}

\textbf{Interpretation \& Communication (20\%):}
\begin{itemize}
    \item Clear agricultural context
    \item Correct statistical interpretation
    \item Practical significance discussion
    \item Professional presentation
\end{itemize}

\textbf{Scientific Rigor (10\%):}
\begin{itemize}
    \item Proper hypothesis formation
    \item Acknowledgment of limitations
    \item Evidence-based conclusions
    \item Reproducible methodology
\end{itemize}
\end{projectbox}

\section{Agricultural Applications}

\begin{infobox}
\textbf{Potential Research Areas:}
\begin{itemize}
    \item \textbf{Crop Production} - Yield comparisons, variety trials, treatment effects
    \item \textbf{Soil Science} - Nutrient analysis, pH effects, organic matter studies
    \item \textbf{Plant Breeding} - Genetic line comparisons, trait correlations
    \item \textbf{Pest Management} - Treatment efficacy, resistance studies
    \item \textbf{Environmental Impact} - Sustainability metrics, carbon footprint
    \item \textbf{Economic Analysis} - Cost-benefit analysis, profitability studies
    \item \textbf{Quality Assessment} - Product quality, post-harvest analysis
    \item \textbf{Precision Agriculture} - Spatial analysis, technology adoption
\end{itemize}
\end{infobox}

\section{Statistical Methods Reference}

\subsection{Common Agricultural Statistics}

\begin{analysisbox}
\textbf{Descriptive Statistics:}\\
\texttt{mean(data\$variable)}\\
\texttt{sd(data\$variable)}\\
\texttt{quantile(data\$variable)}\\[0.3cm]
\textbf{Group Comparisons:}\\
\texttt{t.test(group1, group2)}\\
\texttt{aov(response \textasciitilde{} treatment, data)}\\
\texttt{TukeyHSD(aov\_model)}\\[0.3cm]
\textbf{Relationships:}\\
\texttt{cor(x, y)}\\
\texttt{lm(y \textasciitilde{} x, data)}\\
\texttt{plot(x, y)}\\[0.3cm]
\textbf{Visualization:}\\
\texttt{ggplot(data, aes(x, y)) + geom\_point()}\\
\texttt{ggplot(data, aes(x, y)) + geom\_boxplot()}
\end{analysisbox}

\section{Tips for Success}

\begin{warningbox}
\textbf{Best Practices:}
\begin{itemize}
    \item \textbf{Start Early} - Allow time for data exploration and analysis
    \item \textbf{Check Assumptions} - Verify test requirements before proceeding
    \item \textbf{Document Everything} - Add comments explaining your reasoning
    \item \textbf{Visualize First} - Always plot data before statistical testing
    \item \textbf{Interpret Practically} - Consider agricultural significance
    \item \textbf{Save Frequently} - Download work regularly to avoid loss
    \item \textbf{Test Code} - Run cells in order and check outputs
    \item \textbf{Seek Help Early} - Use office hours for guidance
\end{itemize}
\end{warningbox}

\section{Common Pitfalls to Avoid}

\begin{warningbox}
\textbf{Avoid These Mistakes:}
\begin{itemize}
    \item Using inappropriate statistical tests for data type
    \item Ignoring test assumptions (normality, independence)
    \item Over-interpreting non-significant results
    \item Confusing statistical and practical significance
    \item Poor data visualization choices
    \item Inadequate description of methods and data
    \item Missing or incorrect interpretation of results
    \item Failing to save work before closing Binder
\end{itemize}
\end{warningbox}

\section{Troubleshooting}

\subsection{Technical Issues}

\begin{infobox}
\textbf{Common Problems and Solutions:}
\begin{itemize}
    \item \textbf{Binder won't load:} Refresh page, clear browser cache
    \item \textbf{Package not found:} Use \texttt{install.packages("package\_name")}
    \item \textbf{Code error:} Check syntax, run cells in order
    \item \textbf{Data won't load:} Verify file name and format
    \item \textbf{Lost work:} Always download files before closing
    \item \textbf{Plot not showing:} Check ggplot syntax and data
\end{itemize}
\end{infobox}

\section{Learning Objectives}

By completing this final project, you will demonstrate ability to:
\begin{itemize}
    \item Import and manage agricultural datasets in R
    \item Conduct comprehensive exploratory data analysis
    \item Select and apply appropriate statistical tests
    \item Create professional data visualizations
    \item Interpret statistical results in agricultural context
    \item Communicate findings effectively to stakeholders
    \item Use R programming for reproducible research
    \item Make evidence-based recommendations for agricultural practices
\end{itemize}

\section{Need Help?}

\begin{infobox}
\textbf{Mohammadreza Narimani}\\
Email: mnarimani@ucdavis.edu\\
Department of Biological and Agricultural Engineering, UC Davis\\
Office Hours: Thursdays 10 AM - 12 PM (Zoom)\\
Zoom Link: \href{https://ucdavis.zoom.us/j/99533096447}{Join Office Hours}\\[0.5cm]
\textbf{Additional Resources:}\\
• Course website for all materials\\
• Canvas discussion forum for peer help\\
• R documentation: \texttt{?function\_name}\\
• Online R tutorials and Stack Overflow
\end{infobox}

\vfill

\begin{center}
\textit{Last updated: November 2024 | PLS 120 - Applied Statistics in Agriculture | UC Davis}\\
\textit{Final Project: Complete Statistical Analysis}
\end{center}

\end{document}