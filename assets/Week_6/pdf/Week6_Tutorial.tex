\documentclass[11pt,a4paper]{article}
\usepackage[utf8]{inputenc}
\usepackage[margin=1in]{geometry}
\usepackage{graphicx}
\usepackage[hidelinks]{hyperref}
\usepackage{xcolor}
\usepackage{fancyhdr}
\usepackage{titlesec}
\usepackage{enumitem}
\usepackage{tcolorbox}
\usepackage{fontawesome5}
\usepackage{amsmath}
\usepackage{amssymb}
\usepackage{lmodern}
\usepackage[T1]{fontenc}
\usepackage{needspace}

% Colors
\definecolor{primarygreen}{RGB}{46,125,50}
\definecolor{accentgreen}{RGB}{76,175,80}
\definecolor{lightgray}{RGB}{248,249,250}
\definecolor{bluestat}{RGB}{59,130,246}
\definecolor{redstat}{RGB}{239,68,68}
\definecolor{purplestat}{RGB}{147,51,234}

% Header and footer
\setlength{\headheight}{15pt}
\addtolength{\topmargin}{-3pt}
\pagestyle{fancy}
\fancyhf{}
\fancyhead[L]{\textcolor{primarygreen}{\textbf{PLS 120 - Week 6 Tutorial}}}
\fancyhead[R]{\textcolor{primarygreen}{UC Davis}}
\fancyfoot[C]{\thepage}

% Title formatting
\titleformat{\section}{\large\bfseries\color{primarygreen}}{}{0em}{}[\titlerule]
\titleformat{\subsection}{\normalsize\bfseries\color{primarygreen}}{}{0em}{}

% Custom boxes
\newtcolorbox{infobox}{
    colback=lightgray,
    colframe=primarygreen,
    boxrule=1pt,
    arc=3pt,
    left=10pt,
    right=10pt,
    top=10pt,
    bottom=10pt
}

\newtcolorbox{warningbox}{
    colback=accentgreen!10,
    colframe=primarygreen,
    boxrule=1pt,
    arc=3pt,
    left=10pt,
    right=10pt,
    top=10pt,
    bottom=10pt
}

\newtcolorbox{formulabox}{
    colback=bluestat!10,
    colframe=bluestat,
    boxrule=1pt,
    arc=3pt,
    left=10pt,
    right=10pt,
    top=10pt,
    bottom=10pt
}

\newtcolorbox{ttestbox}{
    colback=purplestat!10,
    colframe=purplestat,
    boxrule=1pt,
    arc=3pt,
    left=10pt,
    right=10pt,
    top=10pt,
    bottom=10pt
}

\begin{document}

% Title page
\begin{titlepage}
    \centering
    \vspace*{2cm}
    
    {\Huge\bfseries\color{primarygreen} PLS 120: Applied Statistics in Agricultural Sciences}
    
    \vspace{1cm}
    
    {\Large\color{primarygreen} Confidence Intervals and T-Tests}
    
    \vspace{2cm}
    
    \includegraphics[width=0.3\textwidth]{../../images/logos/Home_Page_Logo.png}
    
    \vspace{2cm}
    
    {\large\bfseries Week 6 Tutorial Guide}
    
    \vspace{1cm}
    
    {\large Mohammadreza Narimani}\\
    {\normalsize Department of Biological and Agricultural Engineering}\\
    {\normalsize University of California, Davis}
    
    \vspace{1cm}
    
    {\normalsize mnarimani@ucdavis.edu}
    
    \vfill
    
    {\normalsize October 2025}
\end{titlepage}

{\small \tableofcontents}

\section{Important Links}

\begin{tcolorbox}[colback=accentgreen!20, colframe=primarygreen, boxrule=2pt, arc=5pt, title={\textbf{\Large Essential Course Resources}}]
\centering
\textbf{\Large Course Website}\\[0.5cm]
\textcolor{primarygreen}{\textbf{All course materials available at:}}\\[0.3cm]
\href{https://mohammadrezanarimaniucdavis.github.io/PLS120-Statistics-Lab-Materials/}{\textcolor{primarygreen}{\underline{Course Website Link}}}\\[0.8cm]

\textbf{\Large Interactive Binder Environment}\\[0.5cm]
\textcolor{primarygreen}{\textbf{Access Week 6 lab materials:}}\\[0.3cm]
\href{https://mybinder.org/v2/gh/MohammadrezaNarimaniUCDavis/PLS120-Statistics-Lab-Materials/binder-week6}{\textcolor{primarygreen}{\underline{Week 6 Binder Link}}}
\end{tcolorbox}

\section{Welcome to Week 6: Confidence Intervals and T-Tests}

This week, we dive into \textbf{t-distribution and hypothesis testing} - essential tools for agricultural research when population parameters are unknown. You'll learn when to use t-tests instead of z-tests, construct t-based confidence intervals, and analyze treatment effectiveness!

\section{T-Distribution vs Z-Distribution}

\subsection{When to Use T-Distribution}

The t-distribution is used when the population standard deviation is unknown and must be estimated from sample data.

\subsubsection{Key Differences}

\begin{formulabox}
\textbf{Use T-Distribution When:}\\
• Population standard deviation ($\sigma$) is unknown\\
• Using sample standard deviation (s) as estimate\\
• Small to moderate sample sizes (especially n < 30)\\
• More conservative than z-distribution\\[0.3cm]
\textbf{Use Z-Distribution When:}\\
• Population standard deviation ($\sigma$) is known\\
• Large sample sizes (n $\geq$ 30) with CLT\\
• Population is normally distributed\\[0.3cm]
\textbf{Key Properties:}\\
• T-distribution has heavier tails than normal\\
• As df increases, t approaches standard normal\\
• More uncertainty = wider confidence intervals
\end{formulabox}

\needspace{5\baselineskip}
\section{Degrees of Freedom}

\subsection{Understanding Degrees of Freedom}

Degrees of freedom (df) represent the number of independent pieces of information available to estimate a parameter.

\subsubsection{Degrees of Freedom Formula}

\begin{formulabox}
\textbf{One-Sample T-Test:}
$$df = n - 1$$
Where:\\
\texttt{n} = sample size\\
\texttt{df} = degrees of freedom\\[0.3cm]
\textbf{Why n-1?}\\
• We lose one degree of freedom when estimating the mean\\
• With n observations and known mean, only (n-1) are free to vary\\
• Smaller df = wider t-distribution = more conservative results\\[0.3cm]
\textbf{R Implementation:}\\
\texttt{df <- length(data) - 1}\\
\texttt{t\_critical <- qt(0.975, df)} \# For 95\% CI
\end{formulabox}

\needspace{5\baselineskip}
\section{T-Based Confidence Intervals}

\subsection{Confidence Interval Formula}

T-based confidence intervals account for the additional uncertainty when estimating population standard deviation.

\subsubsection{T-Confidence Interval Formula}

\begin{ttestbox}
\textbf{T-Based Confidence Interval:}
$$CI = \overline{x} \pm t_{\alpha/2,df} \times \frac{s}{\sqrt{n}}$$
Where:\\
\texttt{$\overline{x}$} = sample mean\\
\texttt{$t_{\alpha/2,df}$} = critical t-value\\
\texttt{s} = sample standard deviation\\
\texttt{n} = sample size\\
\texttt{df} = n - 1\\[0.3cm]
\textbf{R Implementation:}\\
\texttt{sample\_mean <- mean(data)}\\
\texttt{sample\_sd <- sd(data)}\\
\texttt{n <- length(data)}\\
\texttt{df <- n - 1}\\
\texttt{t\_critical <- qt(0.975, df)} \# 95\% CI\\
\texttt{margin\_error <- t\_critical * (sample\_sd / sqrt(n))}\\
\texttt{ci\_lower <- sample\_mean - margin\_error}\\
\texttt{ci\_upper <- sample\_mean + margin\_error}
\end{ttestbox}

\subsection{Comparison: T vs Z Intervals}

\subsubsection{Width Comparison}

\begin{ttestbox}
\textbf{Interval Width Comparison:}\\
• T-based intervals are WIDER than z-based intervals\\
• Difference is larger for smaller sample sizes\\
• As n increases, t-intervals approach z-intervals\\[0.3cm]
\textbf{Example (n=10, 95\% confidence):}\\
Z-critical value: 1.96\\
T-critical value: 2.262\\
T-interval is about 15\% wider\\[0.3cm]
\textbf{Example (n=30, 95\% confidence):}\\
Z-critical value: 1.96\\
T-critical value: 2.045\\
T-interval is about 4\% wider
\end{ttestbox}

\needspace{5\baselineskip}
\section{One-Sample T-Test}

\subsection{Hypothesis Testing with T-Tests}

T-tests allow us to test hypotheses about population means when the population standard deviation is unknown.

\subsubsection{One-Sample T-Test Formula}

\begin{ttestbox}
\textbf{T-Test Statistic:}
$$t = \frac{\overline{x} - \mu_0}{\frac{s}{\sqrt{n}}}$$
Where:\\
\texttt{$\overline{x}$} = sample mean\\
\texttt{$\mu_0$} = hypothesized population mean\\
\texttt{s} = sample standard deviation\\
\texttt{n} = sample size\\[0.3cm]
\textbf{Decision Rule:}\\
If $|t| > t_{\alpha/2,df}$, reject $H_0$\\
If p-value < $\alpha$, reject $H_0$\\[0.3cm]
\textbf{R Implementation:}\\
\texttt{t.test(data, mu = hypothesized\_mean)}\\
\texttt{t.test(data, mu = 60, conf.level = 0.95)}
\end{ttestbox}

\subsection{Interpreting T-Test Results}

\subsubsection{Understanding Output}

\begin{ttestbox}
\textbf{T-Test Output Components:}\\
• \textbf{t-statistic:} How many standard errors the sample mean is from $\mu_0$\\
• \textbf{degrees of freedom:} n - 1\\
• \textbf{p-value:} Probability of observing this result if $H_0$ is true\\
• \textbf{confidence interval:} Range of plausible values for $\mu$\\[0.3cm]
\textbf{Agricultural Example:}\\
Testing if new fertilizer increases yield above 60 bu/acre\\
$H_0: \mu = 60$ vs $H_a: \mu > 60$\\
If t = 2.34, df = 24, p = 0.028\\
Conclusion: Reject $H_0$, fertilizer significantly increases yield
\end{ttestbox}

\needspace{5\baselineskip}
\section{Assignment 6 Overview}

\subsection{Assignment Structure (20 points total)}

\begin{enumerate}
    \item \textbf{Part 1: Overall Wheat Yield Analysis (9 points)}
    \begin{itemize}
        \item Load and explore wheat yield dataset (1 point)
        \item Calculate sample size and basic statistics (1 point)
        \item Compute mean and standard deviation (2 points)
        \item Calculate z-score for specific value (1 point)
        \item Construct 95\% confidence interval (3 points)
        \item Interpret results in agricultural context (1 point)
    \end{itemize}
    
    \item \textbf{Part 2: Treatment Comparison (11 points)}
    \begin{itemize}
        \item Separate control and fertilizer groups (1 point)
        \item Calculate treatment statistics (2 points)
        \item Compute standard errors (2 points)
        \item Calculate margins of error (2 points)
        \item Construct confidence intervals for each treatment (3 points)
        \item Compare treatments and interpret results (1 point)
    \end{itemize}
\end{enumerate}

\needspace{5\baselineskip}
\section{Agricultural Applications}

\begin{infobox}
\textbf{Real-World T-Test Applications:}
\begin{itemize}
    \item \textbf{Fertilizer Effectiveness} - Test if new fertilizer significantly increases crop yield
    \item \textbf{Variety Trials} - Compare new crop varieties against established standards
    \item \textbf{Treatment Efficacy} - Evaluate pesticide or herbicide effectiveness
    \item \textbf{Quality Control} - Test if product meets quality standards
    \item \textbf{Environmental Impact} - Assess effects of farming practices on soil health
    \item \textbf{Breeding Programs} - Compare performance of new genetic lines
    \item \textbf{Irrigation Studies} - Test optimal water application rates
    \item \textbf{Harvest Timing} - Determine optimal harvest dates for maximum yield
\end{itemize}
\end{infobox}

\section{Key Concepts Summary}

\subsection{T-Test Fundamentals}

\begin{ttestbox}
\textbf{When to Use T-Tests:}\\
• Population $\sigma$ is unknown (most real situations)\\
• Sample size is small to moderate\\
• Data is approximately normally distributed\\
• Testing hypotheses about means\\[0.3cm]
\textbf{T-Test Assumptions:}\\
• Random sampling from population\\
• Observations are independent\\
• Data is approximately normally distributed\\
• For small samples, normality is more critical\\[0.3cm]
\textbf{Confidence Interval Interpretation:}\\
• 95\% CI: We're 95\% confident the true mean lies in this range\\
• Wider intervals = more uncertainty\\
• T-intervals are more conservative than z-intervals
\end{ttestbox}

\subsection{Statistical Decision Making}

\begin{ttestbox}
\textbf{Hypothesis Testing Steps:}\\
1. State null and alternative hypotheses\\
2. Choose significance level ($\alpha$)\\
3. Calculate test statistic\\
4. Find p-value or critical value\\
5. Make decision and interpret in context\\[0.3cm]
\textbf{Type I and Type II Errors:}\\
Type I Error: Rejecting true $H_0$ (false positive)\\
Type II Error: Failing to reject false $H_0$ (false negative)\\[0.3cm]
\textbf{Practical vs Statistical Significance:}\\
• Statistical significance: p < $\alpha$\\
• Practical significance: Effect size matters in real world\\
• Large samples can detect tiny, unimportant differences
\end{ttestbox}

\section{Data Analysis Workflow}

\subsection{Step-by-Step Analysis}

\begin{enumerate}
    \item \textbf{Data Exploration}
    \begin{itemize}
        \item Load data and examine structure
        \item Check for missing values and outliers
        \item Create summary statistics and visualizations
    \end{itemize}
    
    \item \textbf{Assumption Checking}
    \begin{itemize}
        \item Assess normality (histograms, Q-Q plots)
        \item Check for independence
        \item Identify potential issues
    \end{itemize}
    
    \item \textbf{Statistical Analysis}
    \begin{itemize}
        \item Calculate appropriate test statistics
        \item Construct confidence intervals
        \item Perform hypothesis tests
    \end{itemize}
    
    \item \textbf{Interpretation}
    \begin{itemize}
        \item Interpret results in agricultural context
        \item Consider practical significance
        \item Make recommendations based on findings
    \end{itemize}
\end{enumerate}

\section{Getting Started}

\begin{enumerate}
    \item Launch Week 6 Binder environment
    \item Navigate to \texttt{class\_activity} folder
    \item Open \texttt{Week6\_Confidence\_Intervals.ipynb}
    \item Work through interactive exercises
    \item Complete Assignment 6 in \texttt{assignment} folder
\end{enumerate}

\section{Learning Objectives}

By the end of this week, you will be able to:
\begin{itemize}
    \item Distinguish between t-distribution and z-distribution applications
    \item Calculate degrees of freedom for one-sample t-tests
    \item Construct t-based confidence intervals
    \item Perform one-sample t-tests for hypothesis testing
    \item Compare treatment groups using confidence intervals
    \item Interpret t-test results in agricultural contexts
    \item Understand the relationship between sample size and interval width
    \item Make data-driven decisions about treatment effectiveness
\end{itemize}

\section{Tips for Success}

\begin{warningbox}
\textbf{Best Practices:}
\begin{itemize}
    \item Always check if you should use t or z distribution
    \item Remember: df = n - 1 for one-sample t-tests
    \item T-intervals are wider than z-intervals (more conservative)
    \item Use \texttt{t.test()} function in R for complete analysis
    \item Check normality assumptions, especially for small samples
    \item Interpret confidence intervals correctly (about the parameter, not individual observations)
    \item Consider both statistical and practical significance
    \item Always interpret results in the context of the agricultural problem
\end{itemize}
\end{warningbox}

\section{Common Mistakes to Avoid}

\begin{warningbox}
\textbf{Avoid These Errors:}
\begin{itemize}
    \item Using z-distribution when $\sigma$ is unknown
    \item Forgetting to subtract 1 for degrees of freedom
    \item Misinterpreting confidence intervals
    \item Ignoring assumptions (normality, independence)
    \item Confusing statistical significance with practical importance
    \item Not considering the agricultural context in interpretations
    \item Using wrong tail for one-sided vs two-sided tests
\end{itemize}
\end{warningbox}

\section{Need Help?}

\begin{infobox}
\textbf{Mohammadreza Narimani}\\
Email: mnarimani@ucdavis.edu\\
Department of Biological and Agricultural Engineering, UC Davis\\
Office Hours: Thursdays 10 AM - 12 PM (Zoom)\\
Zoom Link: \href{https://ucdavis.zoom.us/j/99533096447}{Join Office Hours}
\end{infobox}

\vfill

\begin{center}
\textit{Last updated: October 2025 | PLS 120 - Applied Statistics in Agriculture | UC Davis}\\
\textit{Week 6: Confidence Intervals and T-Tests}
\end{center}

\end{document}