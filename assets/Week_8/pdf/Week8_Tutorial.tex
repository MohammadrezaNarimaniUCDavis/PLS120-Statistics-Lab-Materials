\documentclass[11pt,a4paper]{article}
\usepackage[utf8]{inputenc}
\usepackage[margin=1in]{geometry}
\usepackage{graphicx}
\usepackage[hidelinks]{hyperref}
\usepackage{xcolor}
\usepackage{fancyhdr}
\usepackage{titlesec}
\usepackage{enumitem}
\usepackage{tcolorbox}
\usepackage{fontawesome5}
\usepackage{amsmath}
\usepackage{amssymb}
\usepackage{lmodern}
\usepackage[T1]{fontenc}
\usepackage{needspace}

% Colors
\definecolor{primarygreen}{RGB}{46,125,50}
\definecolor{accentgreen}{RGB}{76,175,80}
\definecolor{lightgray}{RGB}{248,249,250}
\definecolor{bluestat}{RGB}{59,130,246}
\definecolor{redstat}{RGB}{239,68,68}
\definecolor{purplestat}{RGB}{147,51,234}
\definecolor{orangestat}{RGB}{249,115,22}

% Header and footer
\setlength{\headheight}{15pt}
\addtolength{\topmargin}{-3pt}
\pagestyle{fancy}
\fancyhf{}
\fancyhead[L]{\textcolor{primarygreen}{\textbf{PLS 120 - Week 8 Tutorial}}}
\fancyhead[R]{\textcolor{primarygreen}{UC Davis}}
\fancyfoot[C]{\thepage}

% Title formatting
\titleformat{\section}{\large\bfseries\color{primarygreen}}{}{0em}{}[\titlerule]
\titleformat{\subsection}{\normalsize\bfseries\color{primarygreen}}{}{0em}{}

% Custom boxes
\newtcolorbox{infobox}{
    colback=lightgray,
    colframe=primarygreen,
    boxrule=1pt,
    arc=3pt,
    left=10pt,
    right=10pt,
    top=10pt,
    bottom=10pt
}

\newtcolorbox{warningbox}{
    colback=accentgreen!10,
    colframe=primarygreen,
    boxrule=1pt,
    arc=3pt,
    left=10pt,
    right=10pt,
    top=10pt,
    bottom=10pt
}

\newtcolorbox{formulabox}{
    colback=bluestat!10,
    colframe=bluestat,
    boxrule=1pt,
    arc=3pt,
    left=10pt,
    right=10pt,
    top=10pt,
    bottom=10pt
}

\newtcolorbox{ttestbox}{
    colback=purplestat!10,
    colframe=purplestat,
    boxrule=1pt,
    arc=3pt,
    left=10pt,
    right=10pt,
    top=10pt,
    bottom=10pt
}

\newtcolorbox{chibox}{
    colback=redstat!10,
    colframe=redstat,
    boxrule=1pt,
    arc=3pt,
    left=10pt,
    right=10pt,
    top=10pt,
    bottom=10pt
}

\begin{document}

% Title page
\begin{titlepage}
    \centering
    \vspace*{2cm}
    
    {\Huge\bfseries\color{primarygreen} PLS 120: Applied Statistics in Agricultural Sciences}
    
    \vspace{1cm}
    
    {\Large\color{primarygreen} Hypothesis Testing and Statistical Analysis}
    
    \vspace{2cm}
    
    \includegraphics[width=0.3\textwidth]{../../images/logos/Home_Page_Logo.png}
    
    \vspace{2cm}
    
    {\large\bfseries Week 8 Tutorial Guide}
    
    \vspace{1cm}
    
    {\large Mohammadreza Narimani}\\
    {\normalsize Department of Biological and Agricultural Engineering}\\
    {\normalsize University of California, Davis}
    
    \vspace{1cm}
    
    {\normalsize mnarimani@ucdavis.edu}
    
    \vfill
    
    {\normalsize November 2024}
\end{titlepage}

{\small \tableofcontents}

\section{Important Links}

\begin{tcolorbox}[colback=accentgreen!20, colframe=primarygreen, boxrule=2pt, arc=5pt, title={\textbf{\Large Essential Course Resources}}]
\centering
\textbf{\Large Course Website}\\[0.5cm]
\textcolor{primarygreen}{\textbf{All course materials available at:}}\\[0.3cm]
\href{https://mohammadrezanarimaniucdavis.github.io/PLS120-Statistics-Lab-Materials/}{\textcolor{primarygreen}{\underline{Course Website Link}}}\\[0.8cm]

\textbf{\Large Interactive Binder Environment}\\[0.5cm]
\textcolor{primarygreen}{\textbf{Access Week 8 lab materials:}}\\[0.3cm]
\href{https://mybinder.org/v2/gh/MohammadrezaNarimaniUCDavis/PLS120-Statistics-Lab-Materials/binder-week8}{\textcolor{primarygreen}{\underline{Week 8 Binder Link}}}
\end{tcolorbox}

\section{Welcome to Week 8: Hypothesis Testing and Statistical Analysis}

This week, we dive deep into \textbf{hypothesis testing methods} - essential skills for making statistical inferences in agricultural and biological research. You'll learn one-sample tests, paired t-tests, and chi-square analysis for comprehensive data analysis!

\section{One-Sample T-Tests}

\subsection{When to Use One-Sample T-Tests}

One-sample t-tests compare a sample mean against a known or hypothesized population value, such as testing if crop yield meets a target standard.

\subsubsection{One-Sample T-Test Formula}

\begin{formulabox}
\textbf{One-Sample T-Test:}
$$t = \frac{\overline{x} - \mu_0}{s/\sqrt{n}}$$
Where:\\
\texttt{$\overline{x}$} = sample mean\\
\texttt{$\mu_0$} = hypothesized population mean\\
\texttt{$s$} = sample standard deviation\\
\texttt{$n$} = sample size\\[0.3cm]
\textbf{Degrees of Freedom:} $df = n - 1$\\[0.3cm]
\textbf{R Implementation:}\\
\texttt{t.test(data, mu = hypothesized\_value)}\\
\texttt{t.test(data, mu = 4, alternative = "two.sided")}\\
\texttt{t.test(data, mu = 4, alternative = "greater")}\\
\texttt{t.test(data, mu = 4, alternative = "less")}
\end{formulabox}

\section{Two-Sample T-Tests (Independent and Paired)}

\subsection{Independent Two-Sample T-Tests}

Compare means between two independent groups using Welch's t-test for unequal variances.

\subsubsection{Independent T-Test Formula}

\begin{ttestbox}
\textbf{Two-Sample T-Test (Welch's):}
$$t = \frac{\overline{x_1} - \overline{x_2}}{\sqrt{\frac{s_1^2}{n_1} + \frac{s_2^2}{n_2}}}$$
Where:\\
\texttt{$\overline{x_1}, \overline{x_2}$} = sample means\\
\texttt{$s_1, s_2$} = sample standard deviations\\
\texttt{$n_1, n_2$} = sample sizes\\[0.3cm]
\textbf{R Implementation:}\\
\texttt{t.test(group1, group2, var.equal = FALSE)}\\
\texttt{t.test(group1, group2, alternative = "two.sided")}
\end{ttestbox}

\subsection{Paired T-Tests}

For before/after measurements or matched pairs where observations are dependent.

\subsubsection{Paired T-Test Formula}

\begin{ttestbox}
\textbf{Paired T-Test:}
$$t = \frac{\overline{d}}{s_d/\sqrt{n}}$$
Where:\\
\texttt{$\overline{d}$} = mean of differences\\
\texttt{$s_d$} = standard deviation of differences\\
\texttt{$n$} = number of pairs\\[0.3cm]
\textbf{Agricultural Example:}\\
Biodiversity before and after wildfire on same plots\\[0.3cm]
\textbf{R Implementation:}\\
\texttt{t.test(before, after, paired = TRUE)}\\
\texttt{t.test(before, after, paired = TRUE, alternative = "less")}
\end{ttestbox}

\needspace{5\baselineskip}
\section{Hypothesis Testing Methods}

\subsection{Three Approaches to Hypothesis Testing}

All methods give the same conclusion but offer different perspectives.

\subsubsection{Method Comparison}

\begin{formulabox}
\textbf{1. Critical Value Method:}\\
Compare $|t_{stat}|$ with $t_{critical}$\\
If $|t_{stat}| > t_{critical}$ → Reject $H_0$\\[0.3cm]
\textbf{2. P-Value Method:}\\
Compare p-value with $\alpha$ (significance level)\\
If p-value $< \alpha$ → Reject $H_0$\\[0.3cm]
\textbf{3. R t.test() Function:}\\
Automatically calculates test statistic, p-value, and confidence interval\\
Provides complete statistical output\\[0.3cm]
\textbf{P-Value Calculation (Two-tailed):}
$$p = 2 \times P(T > |t_{stat}|)$$
\textbf{Critical Value:}
$$t_{critical} = t_{\alpha/2, df}$$
\end{formulabox}

\section{Chi-Square Tests}

\subsection{Chi-Square Test Applications}

Chi-square tests analyze categorical data for goodness of fit and independence.

\subsubsection{Chi-Square Formulas}

\begin{chibox}
\textbf{Chi-Square Test Statistic:}
$$\chi^2 = \sum \frac{(O - E)^2}{E}$$
Where:\\
\texttt{O} = observed frequency\\
\texttt{E} = expected frequency\\[0.3cm]
\textbf{Goodness of Fit Test:}\\
Tests if observed frequencies match expected distribution\\
Example: Testing if dice is fair\\
$df = k - 1$ (k = number of categories)\\[0.3cm]
\textbf{Test of Independence:}\\
Tests if two categorical variables are independent\\
Example: Gender vs fruit preference\\
$df = (r-1)(c-1)$ (r = rows, c = columns)\\[0.3cm]
\textbf{R Implementation:}\\
\texttt{chisq.test(observed, p = expected\_proportions)}\\
\texttt{chisq.test(contingency\_table)}
\end{chibox}

\needspace{5\baselineskip}
\section{Assignment 8 Overview}

\subsection{Assignment Structure (15 points total)}

\begin{enumerate}
    \item \textbf{Part 1: Data Import and Visualization (3 points)}
    \begin{itemize}
        \item Load biodiversity data and check distributions (1 point)
        \item Create boxplots and interpret patterns (2 points)
    \end{itemize}
    
    \item \textbf{Part 2: Hypothesis Testing (9 points)}
    \begin{itemize}
        \item Test if wildfire significantly affected biodiversity (3 points)
        \item Test if wildfire significantly increased biodiversity (3 points)
        \item Test if wildfire significantly reduced biodiversity (3 points)
    \end{itemize}
    
    \item \textbf{Part 3: Analysis Limitations (3 points)}
    \begin{itemize}
        \item Discuss limitations of statistical analysis approach (3 points)
    \end{itemize}
\end{enumerate}

\needspace{5\baselineskip}
\section{Environmental Applications}

\begin{infobox}
\textbf{Real-World Hypothesis Testing Applications:}
\begin{itemize}
    \item \textbf{Wildfire Impact Studies} - Assess ecological effects using before/after data
    \item \textbf{Treatment Effectiveness} - Test agricultural interventions with paired designs
    \item \textbf{Biodiversity Research} - Compare species richness across conditions
    \item \textbf{Climate Change Studies} - Analyze environmental parameter changes
    \item \textbf{Conservation Evaluation} - Test effectiveness of protection measures
    \item \textbf{Soil Health Assessment} - Compare management practices effects
    \item \textbf{Water Quality Monitoring} - Test for changes in contamination levels
    \item \textbf{Crop Performance} - Evaluate variety trials and treatment effects
\end{itemize}
\end{infobox}

\section{Hypothesis Formation Guidelines}

\subsection{Proper Hypothesis Setup}

\subsubsection{Hypothesis Types and Examples}

\begin{ttestbox}
\textbf{Two-Sided Hypotheses:}\\
$H_0: \mu_{before} = \mu_{after}$ (no change)\\
$H_1: \mu_{before} \ne \mu_{after}$ (any change)\\
Use when: Testing for any effect (increase or decrease)\\[0.3cm]
\textbf{One-Sided Hypotheses (Greater):}\\
$H_0: \mu_{before} \leq \mu_{after}$ (no increase)\\
$H_1: \mu_{before} > \mu_{after}$ (increase occurred)\\
Use when: Testing for specific directional effect\\[0.3cm]
\textbf{One-Sided Hypotheses (Less):}\\
$H_0: \mu_{before} \ge \mu_{after}$ (no decrease)\\
$H_1: \mu_{before} < \mu_{after}$ (decrease occurred)\\
Use when: Testing for reduction or decline\\[0.3cm]
\textbf{Environmental Example:}\\
"Has wildfire significantly reduced biodiversity?"\\
$H_0$: Biodiversity unchanged or increased\\
$H_1$: Biodiversity decreased after wildfire
\end{ttestbox}

\section{Statistical Interpretation}

\subsection{Understanding P-Values and Confidence Intervals}

\subsubsection{Interpretation Guidelines}

\begin{formulabox}
\textbf{P-Value Interpretation:}\\
p-value = Probability of observing test statistic this extreme\\
or more extreme, assuming $H_0$ is true\\[0.3cm]
\textbf{Decision Rules:}\\
If p < $\alpha$ (0.05) → Reject $H_0$ (significant result)\\
If p $\ge$ $\alpha$ (0.05) → Fail to reject $H_0$ (not significant)\\[0.3cm]
\textbf{Confidence Interval Interpretation:}\\
95\% CI for difference: We are 95\% confident the true\\
difference lies within this interval\\
If CI excludes 0 → Significant difference\\
If CI includes 0 → No significant difference\\[0.3cm]
\textbf{Effect Size Considerations:}\\
Statistical significance $\ne$ Practical importance\\
Consider magnitude of difference in real-world context\\
Large samples can detect tiny, meaningless differences
\end{formulabox}

\section{Data Analysis Workflow}

\subsection{Step-by-Step Hypothesis Testing}

\begin{enumerate}
    \item \textbf{Data Exploration}
    \begin{itemize}
        \item Load and examine data structure
        \item Check for missing values and outliers
        \item Create visualizations (histograms, boxplots)
        \item Assess normality assumptions
    \end{itemize}
    
    \item \textbf{Hypothesis Formation}
    \begin{itemize}
        \item Define research question clearly
        \item State null and alternative hypotheses
        \item Choose test type (one/two-sided)
        \item Set significance level ($\alpha = 0.05$)
    \end{itemize}
    
    \item \textbf{Test Selection and Execution}
    \begin{itemize}
        \item Choose appropriate test (one-sample, paired, independent)
        \item Verify test assumptions
        \item Perform statistical test using R
        \item Calculate effect sizes if needed
    \end{itemize}
    
    \item \textbf{Results Interpretation}
    \begin{itemize}
        \item Interpret p-values and confidence intervals
        \item Consider practical significance
        \item Draw conclusions in context
        \item Discuss limitations and assumptions
    \end{itemize}
\end{enumerate}

\section{Key Concepts Summary}

\subsection{Test Selection Guide}

\begin{formulabox}
\textbf{When to Use Each Test:}\\[0.3cm]
\textbf{One-Sample T-Test:}\\
• Compare sample mean to known standard\\
• Example: Does yield meet target of 60 bu/acre?\\[0.3cm]
\textbf{Independent Two-Sample T-Test:}\\
• Compare means between unrelated groups\\
• Example: Organic vs conventional farming yields\\[0.3cm]
\textbf{Paired T-Test:}\\
• Compare before/after on same subjects\\
• Example: Biodiversity before/after treatment\\[0.3cm]
\textbf{Chi-Square Test:}\\
• Analyze categorical data\\
• Example: Disease presence across varieties\\[0.3cm]
\textbf{Test Assumptions:}\\
• Independence of observations\\
• Approximate normality (t-tests)\\
• Expected frequencies $\ge$ 5 (chi-square)
\end{formulabox}

\section{Common Analysis Limitations}

\subsection{Statistical Analysis Constraints}

\begin{warningbox}
\textbf{Important Limitations to Consider:}
\begin{itemize}
    \item \textbf{Sample Size} - Small samples reduce power to detect effects
    \item \textbf{Data Quality} - Measurement errors affect reliability
    \item \textbf{Time Window} - Limited observation period may miss effects
    \item \textbf{Single Parameter Focus} - Relying only on means ignores variability
    \item \textbf{Normality Assumptions} - Violations can affect test validity
    \item \textbf{Independence Assumptions} - Spatial/temporal correlation issues
    \item \textbf{Multiple Testing} - Increased Type I error with many tests
    \item \textbf{Confounding Variables} - Unmeasured factors affecting results
    \item \textbf{Generalizability} - Results may not apply to other contexts
    \item \textbf{Causation vs Correlation} - Statistical association $\ne$ causation
\end{itemize}
\end{warningbox}

\section{Getting Started}

\begin{enumerate}
    \item Launch Week 8 Binder environment
    \item Navigate to \texttt{class\_activity} folder
    \item Open \texttt{Week8\_Correlation\_Analysis.ipynb}
    \item Work through hypothesis testing examples
    \item Complete Assignment 8 in \texttt{assignment} folder
\end{enumerate}

\section{Learning Objectives}

By the end of this week, you will be able to:
\begin{itemize}
    \item Form appropriate null and alternative hypotheses
    \item Perform one-sample t-tests using multiple methods
    \item Conduct two-sample and paired t-tests appropriately
    \item Apply chi-square tests for categorical data analysis
    \item Interpret statistical results and draw valid conclusions
    \item Understand limitations of statistical analyses
    \item Choose appropriate tests for different research questions
    \item Apply hypothesis testing to environmental research problems
\end{itemize}

\section{Tips for Success}

\begin{warningbox}
\textbf{Best Practices:}
\begin{itemize}
    \item Check test assumptions before analysis (normality, independence)
    \item Visualize data first using histograms and boxplots
    \item Choose appropriate test type based on research question
    \item Interpret results carefully - statistical $\ne$ practical significance
    \item Consider limitations and discuss them in conclusions
    \item Use paired tests for before/after or matched data
    \item Plan hypothesis tests before seeing data
    \item Always interpret results in scientific context
\end{itemize}
\end{warningbox}

\section{Common Mistakes to Avoid}

\begin{warningbox}
\textbf{Avoid These Errors:}
\begin{itemize}
    \item Confusing paired and independent t-tests
    \item Misinterpreting p-values as probability hypotheses are true
    \item Using wrong alternative hypothesis direction
    \item Ignoring test assumptions (normality, independence)
    \item Over-interpreting non-significant results
    \item Switching test type after seeing results
    \item Forgetting to consider practical significance
    \item Not discussing analysis limitations
    \item Using inappropriate tests for data type
    \item Making causal claims from correlational data
\end{itemize}
\end{warningbox}

\section{Need Help?}

\begin{infobox}
\textbf{Mohammadreza Narimani}\\
Email: mnarimani@ucdavis.edu\\
Department of Biological and Agricultural Engineering, UC Davis\\
Office Hours: Thursdays 10 AM - 12 PM (Zoom)\\
Zoom Link: \href{https://ucdavis.zoom.us/j/99533096447}{Join Office Hours}
\end{infobox}

\vfill

\begin{center}
\textit{Last updated: November 2024 | PLS 120 - Applied Statistics in Agriculture | UC Davis}\\
\textit{Week 8: Hypothesis Testing and Statistical Analysis}
\end{center}

\end{document}