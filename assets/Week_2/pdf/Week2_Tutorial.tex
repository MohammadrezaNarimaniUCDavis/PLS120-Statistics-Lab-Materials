\documentclass[11pt,a4paper]{article}
\usepackage[utf8]{inputenc}
\usepackage[margin=1in]{geometry}
\usepackage{graphicx}
\usepackage[hidelinks]{hyperref}
\usepackage{xcolor}
\usepackage{fancyhdr}
\usepackage{titlesec}
\usepackage{enumitem}
\usepackage{tcolorbox}
\usepackage{fontawesome5}
\usepackage{amsmath}
\usepackage{amssymb}
\usepackage{lmodern}
\usepackage[T1]{fontenc}

% Colors
\definecolor{primarygreen}{RGB}{46,125,50}
\definecolor{accentgreen}{RGB}{76,175,80}
\definecolor{lightgray}{RGB}{248,249,250}
\definecolor{bluestat}{RGB}{59,130,246}
\definecolor{redstat}{RGB}{239,68,68}

% Header and footer
\setlength{\headheight}{15pt}
\addtolength{\topmargin}{-3pt}
\pagestyle{fancy}
\fancyhf{}
\fancyhead[L]{\textcolor{primarygreen}{\textbf{PLS 120 - Week 2 Tutorial}}}
\fancyhead[R]{\textcolor{primarygreen}{UC Davis}}
\fancyfoot[C]{\thepage}

% Title formatting
\titleformat{\section}{\Large\bfseries\color{primarygreen}}{}{0em}{}[\titlerule]
\titleformat{\subsection}{\large\bfseries\color{primarygreen}}{}{0em}{}

% Custom boxes
\newtcolorbox{infobox}{
    colback=lightgray,
    colframe=primarygreen,
    boxrule=1pt,
    arc=3pt,
    left=10pt,
    right=10pt,
    top=10pt,
    bottom=10pt
}

\newtcolorbox{warningbox}{
    colback=accentgreen!10,
    colframe=primarygreen,
    boxrule=1pt,
    arc=3pt,
    left=10pt,
    right=10pt,
    top=10pt,
    bottom=10pt
}

\newtcolorbox{formulabox}{
    colback=bluestat!10,
    colframe=bluestat,
    boxrule=1pt,
    arc=3pt,
    left=10pt,
    right=10pt,
    top=10pt,
    bottom=10pt
}

\begin{document}

% Title page
\begin{titlepage}
    \centering
    \vspace*{2cm}
    
    {\Huge\bfseries\color{primarygreen} PLS 120: Applied Statistics in Agricultural Sciences}
    
    \vspace{1cm}
    
    {\Large\color{primarygreen} Descriptive Statistics and Central Tendency}
    
    \vspace{2cm}
    
    \includegraphics[width=0.3\textwidth]{../../images/logos/Home_Page_Logo.png}
    
    \vspace{2cm}
    
    {\large\bfseries Week 2 Tutorial Guide}
    
    \vspace{1cm}
    
    {\large Mohammadreza Narimani}\\
    {\normalsize Department of Biological and Agricultural Engineering}\\
    {\normalsize University of California, Davis}
    
    \vspace{1cm}
    
    {\normalsize mnarimani@ucdavis.edu}
    
    \vfill
    
    {\normalsize October 2025}
\end{titlepage}

\tableofcontents
\newpage

\section{Important Links}

\begin{tcolorbox}[colback=accentgreen!20, colframe=primarygreen, boxrule=2pt, arc=5pt, title={\textbf{\Large Essential Course Resources}}]
\centering
\textbf{\Large Course Website}\\[0.5cm]
\textcolor{primarygreen}{\textbf{All course materials are available at:}}\\[0.3cm]
\href{https://mohammadrezanarimaniucdavis.github.io/PLS120-Statistics-Lab-Materials/}{{\textbf{\Large \textcolor{primarygreen}{\underline{Click Here to Access Course Website}}}}}\\[0.8cm]

\textbf{\Large Interactive Binder Environment}\\[0.5cm]
\textcolor{primarygreen}{\textbf{Access Week 2 lab materials directly:}}\\[0.3cm]
\href{https://mybinder.org/v2/gh/MohammadrezaNarimaniUCDavis/PLS120-Statistics-Lab-Materials/binder-week2}{{\textbf{\Large \textcolor{primarygreen}{\underline{Click Here to Launch Week 2 Binder}}}}}
\end{tcolorbox}

\section{Welcome to Week 2: Descriptive Statistics}

This week, we explore \textbf{descriptive statistics} and \textbf{measures of central tendency} - essential tools for summarizing and understanding data in agricultural research. You'll learn to calculate mean, median, mode, variance, and standard deviation using R!

\section{Key Statistical Concepts}

\subsection{Measures of Central Tendency}

Understanding where the "center" of your data lies is fundamental to statistical analysis.

\subsubsection{\texorpdfstring{Mean ($\mu$ or $\bar{x}$)}{Mean (mu or x-bar)}}

\begin{formulabox}
\textbf{Definition:} The arithmetic average of all values\\[0.3cm]
\textbf{Formula:} $\mu = \frac{\Sigma x}{n}$\\[0.3cm]
\textbf{When to use:} Most common measure, but sensitive to outliers\\
\textbf{Example:} Average crop yield across all plots
\end{formulabox}

\subsubsection{Median}

\begin{formulabox}
\textbf{Definition:} The middle value when data is ordered\\[0.3cm]
\textbf{Formula:} Middle value of sorted data\\[0.3cm]
\textbf{When to use:} Better for skewed distributions or data with outliers\\
\textbf{Example:} Typical rainfall amount (not affected by extreme storms)
\end{formulabox}

\subsubsection{Mode}

\begin{formulabox}
\textbf{Definition:} The most frequently occurring value\\[0.3cm]
\textbf{Formula:} Most common value in dataset\\[0.3cm]
\textbf{When to use:} Useful for categorical data or finding the most common observation\\
\textbf{Example:} Most common pest species observed
\end{formulabox}

\subsection{Measures of Variability}

Understanding how spread out your data is from the center.

\subsubsection{\texorpdfstring{Variance ($\sigma^2$)}{Variance (sigma-squared)}}

\begin{formulabox}
\textbf{Definition:} Average squared deviation from the mean\\[0.3cm]
\textbf{Formula:} $\sigma^2 = \frac{\Sigma(x - \mu)^2}{n}$\\[0.3cm]
\textbf{Note:} Units are squared, making interpretation less intuitive
\end{formulabox}

\subsubsection{\texorpdfstring{Standard Deviation ($\sigma$)}{Standard Deviation (sigma)}}

\begin{formulabox}
\textbf{Definition:} Square root of variance - average distance from the mean\\[0.3cm]
\textbf{Formula:} $\sigma = \sqrt{\sigma^2}$\\[0.3cm]
\textbf{Advantage:} Same units as original data, easier to interpret\\
\textbf{Rule of thumb:} In normal distributions, approximately 68\% of data falls within 1$\sigma$ of the mean
\end{formulabox}

\subsubsection{Coefficient of Variation (CV)}

\begin{formulabox}
\textbf{Definition:} Relative variability measure\\[0.3cm]
\textbf{Formula:} $CV = \frac{\sigma}{\mu} \times 100\%$\\[0.3cm]
\textbf{When to use:} Allows comparison of variability across different scales or units\\
\textbf{Example:} Compare consistency of wheat yield (tons/ha) vs. corn yield (bushels/acre)
\end{formulabox}

\section{Key R Functions This Week}

\subsection{Summary Statistics}

\begin{infobox}
\texttt{summary(data)} - Comprehensive summary\\
\texttt{mean(data\$column)} - Calculate mean\\
\texttt{median(data\$column)} - Calculate median\\
\texttt{var(data\$column)} - Calculate variance\\
\texttt{sd(data\$column)} - Calculate standard deviation\\
\texttt{quantile(data\$column)} - Calculate quantiles
\end{infobox}

\subsection{Data Exploration}

\begin{infobox}
\texttt{head(data)} - First 6 rows\\
\texttt{str(data)} - Data structure\\
\texttt{nrow(data)} - Number of rows\\
\texttt{ncol(data)} - Number of columns
\end{infobox}

\subsection{Custom Mode Function}

Since R doesn't have a built-in mode function, we create our own:

\begin{infobox}
\texttt{Mode <- function(x) \{}\\
\texttt{~~ux <- unique(x)}\\
\texttt{~~ux[which.max(tabulate(match(x, ux)))]}\\
\texttt{\}}
\end{infobox}

\section{Assignment 2: Central Tendency Analysis}

\subsection{Assignment Overview (20 points total)}

\begin{enumerate}
    \item \textbf{Part 1: Mean, Median, Mode (7 points)}
    \item \textbf{Part 2: Variance \& Standard Deviation (5 points)}
    \item \textbf{Part 3: Quantiles (1 point)}
    \item \textbf{Written Analysis (7 points)}
\end{enumerate}

The assignment uses the LA dataset to compare statistics between all victims, male victims, and female victims.

\section{Why This Matters in Agriculture}

\begin{infobox}
\textbf{Agricultural Applications:}
\begin{itemize}
    \item \textbf{Crop Yields} - Compare mean yields across varieties
    \item \textbf{Soil Properties} - Understand nutrient variability
    \item \textbf{Weather Patterns} - Analyze rainfall and temperature
    \item \textbf{Pest Populations} - Track abundance changes
    \item \textbf{Quality Control} - Monitor product consistency
\end{itemize}
\end{infobox}

\section{Getting Started}

\begin{enumerate}
    \item Launch Week 2 Binder environment
    \item Navigate to \texttt{class\_activity} folder
    \item Open \texttt{Week2\_Descriptive\_Statistics.ipynb}
    \item Work with the built-in iris dataset
\end{enumerate}

\section{Learning Objectives}

By the end of this week, you will be able to:
\begin{itemize}
    \item Calculate and interpret mean, median, and mode
    \item Understand when to use each measure of central tendency
    \item Compute variance, standard deviation, and CV
    \item Use quantiles to understand data distribution
    \item Compare statistics across different subgroups
\end{itemize}

\section{Need Help?}

\begin{infobox}
\textbf{Mohammadreza Narimani}\\
Email: mnarimani@ucdavis.edu\\
Department of Biological and Agricultural Engineering, UC Davis
\end{infobox}

\vfill

\begin{center}
\textit{Last updated: October 2025 | PLS 120 - Applied Statistics in Agriculture | UC Davis}\\
\textit{Week 2: Descriptive Statistics and Central Tendency}
\end{center}

\end{document}