\documentclass[11pt,a4paper]{article}
\usepackage[utf8]{inputenc}
\usepackage[margin=1in]{geometry}
\usepackage{graphicx}
\usepackage[hidelinks]{hyperref}
\usepackage{xcolor}
\usepackage{fancyhdr}
\usepackage{titlesec}
\usepackage{enumitem}
\usepackage{tcolorbox}
\usepackage{fontawesome5}
\usepackage{amsmath}
\usepackage{amssymb}
\usepackage{lmodern}
\usepackage[T1]{fontenc}
\usepackage{needspace}

% Colors
\definecolor{primarygreen}{RGB}{46,125,50}
\definecolor{accentgreen}{RGB}{76,175,80}
\definecolor{lightgray}{RGB}{248,249,250}
\definecolor{bluestat}{RGB}{59,130,246}
\definecolor{redstat}{RGB}{239,68,68}
\definecolor{purplestat}{RGB}{147,51,234}

% Header and footer
\setlength{\headheight}{15pt}
\addtolength{\topmargin}{-3pt}
\pagestyle{fancy}
\fancyhf{}
\fancyhead[L]{\textcolor{primarygreen}{\textbf{PLS 120 - Week 7 Tutorial}}}
\fancyhead[R]{\textcolor{primarygreen}{UC Davis}}
\fancyfoot[C]{\thepage}

% Title formatting
\titleformat{\section}{\large\bfseries\color{primarygreen}}{}{0em}{}[\titlerule]
\titleformat{\subsection}{\normalsize\bfseries\color{primarygreen}}{}{0em}{}

% Custom boxes
\newtcolorbox{infobox}{
    colback=lightgray,
    colframe=primarygreen,
    boxrule=1pt,
    arc=3pt,
    left=10pt,
    right=10pt,
    top=10pt,
    bottom=10pt
}

\newtcolorbox{warningbox}{
    colback=accentgreen!10,
    colframe=primarygreen,
    boxrule=1pt,
    arc=3pt,
    left=10pt,
    right=10pt,
    top=10pt,
    bottom=10pt
}

\newtcolorbox{formulabox}{
    colback=bluestat!10,
    colframe=bluestat,
    boxrule=1pt,
    arc=3pt,
    left=10pt,
    right=10pt,
    top=10pt,
    bottom=10pt
}

\newtcolorbox{ttestbox}{
    colback=purplestat!10,
    colframe=purplestat,
    boxrule=1pt,
    arc=3pt,
    left=10pt,
    right=10pt,
    top=10pt,
    bottom=10pt
}

\begin{document}

% Title page
\begin{titlepage}
    \centering
    \vspace*{2cm}
    
    {\Huge\bfseries\color{primarygreen} PLS 120: Applied Statistics in Agricultural Sciences}
    
    \vspace{1cm}
    
    {\Large\color{primarygreen} Functions and T-Tests}
    
    \vspace{2cm}
    
    \includegraphics[width=0.3\textwidth]{../../images/logos/Home_Page_Logo.png}
    
    \vspace{2cm}
    
    {\large\bfseries Week 7 Tutorial Guide}
    
    \vspace{1cm}
    
    {\large Mohammadreza Narimani}\\
    {\normalsize Department of Biological and Agricultural Engineering}\\
    {\normalsize University of California, Davis}
    
    \vspace{1cm}
    
    {\normalsize mnarimani@ucdavis.edu}
    
    \vfill
    
    {\normalsize November 2024}
\end{titlepage}

{\small \tableofcontents}

\section{Important Links}

\begin{tcolorbox}[colback=accentgreen!20, colframe=primarygreen, boxrule=2pt, arc=5pt, title={\textbf{\Large Essential Course Resources}}]
\centering
\textbf{\Large Course Website}\\[0.5cm]
\textcolor{primarygreen}{\textbf{All course materials available at:}}\\[0.3cm]
\href{https://mohammadrezanarimaniucdavis.github.io/PLS120-Statistics-Lab-Materials/}{\textcolor{primarygreen}{\underline{Course Website Link}}}\\[0.8cm]

\textbf{\Large Interactive Binder Environment}\\[0.5cm]
\textcolor{primarygreen}{\textbf{Access Week 7 lab materials:}}\\[0.3cm]
\href{https://mybinder.org/v2/gh/MohammadrezaNarimaniUCDavis/PLS120-Statistics-Lab-Materials/binder-week7}{\textcolor{primarygreen}{\underline{Week 7 Binder Link}}}
\end{tcolorbox}

\section{Welcome to Week 7: Functions and T-Tests}

This week, we explore \textbf{custom functions in R and two-sample t-tests} - essential tools for creating reusable code and comparing groups in agricultural research. You'll learn to write functions, perform statistical comparisons, and interpret results for evidence-based farming decisions!

\section{Writing Functions in R}

\subsection{Why Use Functions?}

Functions in R are powerful tools that simplify repetitive tasks and improve code efficiency. They make your analysis more organized, reduce errors, and allow for easy reuse across projects.

\subsubsection{Function Syntax}

\begin{formulabox}
\textbf{Basic Function Structure:}
\begin{verbatim}
myFunction <- function(arg1, arg2, ...) {
  # Code to execute
  result <- some_calculation
  return(result)
}
\end{verbatim}
\textbf{Components:}\\
• \texttt{myFunction} = Function name\\
• \texttt{arg1, arg2} = Parameters (inputs)\\
• \texttt{return()} = Specifies output\\[0.3cm]
\textbf{Example - Square Function:}
\begin{verbatim}
square <- function(x) {
  result <- x^2
  return(result)
}

# Usage
square(5)  # Returns 25
\end{verbatim}
\end{formulabox}

\subsection{Sample Size Function}

A practical example of function creation for statistical calculations.

\subsubsection{Sample Size Calculation Function}

\begin{formulabox}
\textbf{Sample Size Formula:}
$$n = \frac{z^2 \cdot p \cdot (1-p)}{E^2}$$
Where:\\
\texttt{z} = critical value from normal distribution\\
\texttt{p} = expected prevalence\\
\texttt{E} = desired margin of error\\[0.3cm]
\textbf{R Function Implementation:}
\begin{verbatim}
calculate_sample_size <- function(prev, alpha, margin_error) {
  z <- qnorm(1 - alpha / 2)
  n <- z^2 * prev * (1 - prev) / (margin_error^2)
  required_sample_size <- ceiling(n)
  return(required_sample_size)
}

# Usage
calculate_sample_size(0.1, 0.05, 0.05)  # Returns 139
\end{verbatim}
\end{formulabox}

\needspace{5\baselineskip}
\section{Two-Sample T-Tests}

\subsection{When to Use Two-Sample T-Tests}

Two-sample t-tests compare means between two independent groups, such as treatment vs control in agricultural experiments.

\subsubsection{Welch's T-Test Formula}

\begin{ttestbox}
\textbf{Two-Sample T-Test (Unequal Variances):}
$$t = \frac{\overline{x_1} - \overline{x_2}}{\sqrt{\frac{s_1^2}{n_1} + \frac{s_2^2}{n_2}}}$$
Where:\\
\texttt{$\overline{x_1}, \overline{x_2}$} = sample means\\
\texttt{$s_1, s_2$} = sample standard deviations\\
\texttt{$n_1, n_2$} = sample sizes\\[0.3cm]
\textbf{Degrees of Freedom (Welch's approximation):}
$$df = \frac{(\frac{s_1^2}{n_1} + \frac{s_2^2}{n_2})^2}{\frac{(s_1^2/n_1)^2}{n_1-1} + \frac{(s_2^2/n_2)^2}{n_2-1}}$$
\textbf{R Implementation:}\\
\texttt{t.test(group1, group2, var.equal = FALSE)}
\end{ttestbox}

\subsection{Hypothesis Formation}

Proper hypothesis setup is crucial for meaningful statistical testing.

\subsubsection{Hypothesis Types}

\begin{ttestbox}
\textbf{Two-Sided Test:}\\
$H_0: \mu_1 = \mu_2$ (means are equal)\\
$H_1: \mu_1 \neq \mu_2$ (means are different)\\[0.3cm]
\textbf{One-Sided Tests:}\\
Greater than: $H_1: \mu_1 > \mu_2$\\
Less than: $H_1: \mu_1 < \mu_2$\\[0.3cm]
\textbf{Agricultural Examples:}\\
• Fertilizer vs Control: Does fertilizer increase yield?\\
• Variety A vs B: Which variety performs better?\\
• Irrigation methods: Is drip more efficient than flood?\\[0.3cm]
\textbf{R Implementation:}\\
\texttt{t.test(treatment, control, alternative = "two.sided")}\\
\texttt{t.test(treatment, control, alternative = "greater")}\\
\texttt{t.test(treatment, control, alternative = "less")}
\end{ttestbox}

\needspace{5\baselineskip}
\section{Interpreting T-Test Results}

\subsection{Understanding T-Test Output}

T-test results provide multiple pieces of information for decision-making.

\subsubsection{Key Output Components}

\begin{ttestbox}
\textbf{T-Test Output Elements:}\\
• \textbf{t-statistic:} Magnitude and direction of difference\\
• \textbf{degrees of freedom:} Based on sample sizes and variances\\
• \textbf{p-value:} Probability of observing result if $H_0$ true\\
• \textbf{confidence interval:} Range for true mean difference\\
• \textbf{sample estimates:} Group means\\[0.3cm]
\textbf{Decision Rules:}\\
If p-value < $\alpha$ (0.05) → Reject $H_0$\\
If confidence interval excludes 0 → Significant difference\\[0.3cm]
\textbf{Agricultural Interpretation:}\\
"The fertilizer treatment shows a statistically significant\\
increase in yield compared to control (p = 0.023).\\
The mean difference is 6.5 bu/acre with 95\% CI [1.2, 11.8]."
\end{ttestbox}

\subsection{One-Sided vs Two-Sided Tests}

Understanding when to use directional hypotheses.

\subsubsection{Test Selection Guidelines}

\begin{ttestbox}
\textbf{Two-Sided Test Use When:}\\
• Testing for any difference (better or worse)\\
• Exploratory research\\
• No prior expectation of direction\\[0.3cm]
\textbf{One-Sided Test Use When:}\\
• Strong theoretical reason for direction\\
• Testing if treatment is better (not just different)\\
• More statistical power for detecting effects in predicted direction\\[0.3cm]
\textbf{Agricultural Examples:}\\
Two-sided: "Does variety A differ from variety B?"\\
One-sided: "Does nitrogen fertilizer increase yield above control?"\\[0.3cm]
\textbf{Caution:}\\
One-sided tests should be planned before data collection\\
Don't switch after seeing results!
\end{ttestbox}

\needspace{5\baselineskip}
\section{Assignment 7 Overview}

\subsection{Assignment Structure (20 points total)}

\begin{enumerate}
    \item \textbf{Part 1: Load and Visualize Data (4 points)}
    \begin{itemize}
        \item Load wheat yield dataset and convert factors (2 points)
        \item Create boxplot visualization (2 points)
    \end{itemize}
    
    \item \textbf{Part 2: Form Hypotheses (3 points)}
    \begin{itemize}
        \item State null and alternative hypotheses clearly (3 points)
    \end{itemize}
    
    \item \textbf{Part 3: Perform T-Test (5 points)}
    \begin{itemize}
        \item Separate data by treatment groups (2 points)
        \item Execute two-sample t-test with proper parameters (3 points)
    \end{itemize}
    
    \item \textbf{Part 4: Interpret Results (4 points)}
    \begin{itemize}
        \item Interpret p-value correctly (1 point)
        \item Draw conclusions about fertilizer treatments (2 points)
        \item Make decision about null hypothesis (1 point)
    \end{itemize}
    
    \item \textbf{Part 5: One-Sided Test (4 points)}
    \begin{itemize}
        \item Perform directional t-test (2 points)
        \item Interpret one-sided test results (2 points)
    \end{itemize}
\end{enumerate}

\needspace{5\baselineskip}
\section{Agricultural Applications}

\begin{infobox}
\textbf{Real-World Function and T-Test Applications:}
\begin{itemize}
    \item \textbf{Treatment Efficacy} - Compare fertilizer, pesticide, or irrigation treatments
    \item \textbf{Variety Trials} - Test new crop varieties against established standards
    \item \textbf{Quality Control} - Automated functions for routine quality assessments
    \item \textbf{Experimental Design} - Sample size calculations for field trials
    \item \textbf{Breeding Programs} - Compare genetic lines for yield and quality traits
    \item \textbf{Environmental Studies} - Test effects of farming practices on soil health
    \item \textbf{Economic Analysis} - Compare profitability of different management strategies
    \item \textbf{Precision Agriculture} - Analyze zone-specific treatment effects
\end{itemize}
\end{infobox}

\section{Key Concepts Summary}

\subsection{Function Development}

\begin{formulabox}
\textbf{Function Best Practices:}\\
• Use descriptive function names\\
• Include parameter validation\\
• Add comments explaining purpose\\
• Test with simple examples first\\
• Return meaningful results\\[0.3cm]
\textbf{Common Function Types in Agriculture:}\\
• Statistical calculations (mean, variance, sample size)\\
• Unit conversions (acres to hectares, etc.)\\
• Economic calculations (profit, cost per unit)\\
• Data transformations and cleaning\\[0.3cm]
\textbf{Function Benefits:}\\
• Reduces code duplication\\
• Easier debugging and maintenance\\
• Consistent calculations across analyses\\
• Shareable with colleagues
\end{formulabox}

\subsection{Statistical Testing Fundamentals}

\begin{ttestbox}
\textbf{T-Test Assumptions:}\\
• Independent observations\\
• Approximately normal distributions\\
• Random sampling from populations\\
• For Welch's t-test: unequal variances OK\\[0.3cm]
\textbf{Effect Size Considerations:}\\
• Statistical significance $\neq$ practical importance\\
• Consider magnitude of difference\\
• Economic significance in agricultural context\\
• Confidence intervals show range of plausible effects\\[0.3cm]
\textbf{Multiple Comparisons:}\\
• Be cautious with many t-tests\\
• Consider family-wise error rate\\
• Use appropriate corrections when needed
\end{ttestbox}

\section{Data Analysis Workflow}

\subsection{Step-by-Step Analysis}

\begin{enumerate}
    \item \textbf{Data Preparation}
    \begin{itemize}
        \item Load and examine data structure
        \item Convert categorical variables to factors
        \item Check for missing values and outliers
        \item Create exploratory visualizations
    \end{itemize}
    
    \item \textbf{Hypothesis Formation}
    \begin{itemize}
        \item Define research question clearly
        \item State null and alternative hypotheses
        \item Choose appropriate test type (one/two-sided)
        \item Set significance level
    \end{itemize}
    
    \item \textbf{Statistical Testing}
    \begin{itemize}
        \item Separate data by groups
        \item Check test assumptions
        \item Perform appropriate t-test
        \item Calculate effect sizes
    \end{itemize}
    
    \item \textbf{Interpretation and Communication}
    \begin{itemize}
        \item Interpret results in agricultural context
        \item Consider practical significance
        \item Make evidence-based recommendations
        \item Communicate uncertainty appropriately
    \end{itemize}
\end{enumerate}

\section{Getting Started}

\begin{enumerate}
    \item Launch Week 7 Binder environment
    \item Navigate to \texttt{class\_activity} folder
    \item Open \texttt{Week7\_Regression\_Analysis.ipynb}
    \item Work through function examples and t-test exercises
    \item Complete Assignment 7 in \texttt{assignment} folder
\end{enumerate}

\section{Learning Objectives}

By the end of this week, you will be able to:
\begin{itemize}
    \item Create custom functions with parameters and return values
    \item Build reusable functions for statistical calculations
    \item Form appropriate null and alternative hypotheses
    \item Perform two-sample t-tests using Welch's method
    \item Interpret t-statistics, p-values, and confidence intervals
    \item Distinguish between one-sided and two-sided tests
    \item Apply statistical testing to agricultural research problems
    \item Make data-driven decisions about treatment effectiveness
\end{itemize}

\section{Tips for Success}

\begin{warningbox}
\textbf{Best Practices:}
\begin{itemize}
    \item Test functions with simple examples before complex data
    \item Always check t-test assumptions (normality, independence)
    \item Use \texttt{var.equal = FALSE} for Welch's t-test (safer default)
    \item Interpret p-values correctly (probability under null hypothesis)
    \item Consider both statistical and practical significance
    \item Visualize data before testing (boxplots, histograms)
    \item Choose test type (one/two-sided) before seeing results
    \item Always interpret results in agricultural context
\end{itemize}
\end{warningbox}

\section{Common Mistakes to Avoid}

\begin{warningbox}
\textbf{Avoid These Errors:}
\begin{itemize}
    \item Writing functions without testing them first
    \item Forgetting to use \texttt{return()} in functions
    \item Using one-sided tests without strong justification
    \item Misinterpreting p-values as probability hypotheses are true
    \item Ignoring practical significance when p-value is small
    \item Not checking t-test assumptions
    \item Switching between one/two-sided tests after seeing data
    \item Over-interpreting non-significant results
\end{itemize}
\end{warningbox}

\section{Need Help?}

\begin{infobox}
\textbf{Mohammadreza Narimani}\\
Email: mnarimani@ucdavis.edu\\
Department of Biological and Agricultural Engineering, UC Davis\\
Office Hours: Thursdays 10 AM - 12 PM (Zoom)\\
Zoom Link: \href{https://ucdavis.zoom.us/j/99533096447}{Join Office Hours}
\end{infobox}

\vfill

\begin{center}
\textit{Last updated: November 2024 | PLS 120 - Applied Statistics in Agriculture | UC Davis}\\
\textit{Week 7: Functions and T-Tests}
\end{center}

\end{document}