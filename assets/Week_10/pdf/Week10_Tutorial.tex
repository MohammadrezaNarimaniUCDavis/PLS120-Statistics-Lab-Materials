\documentclass[11pt,a4paper]{article}
\usepackage[utf8]{inputenc}
\usepackage[margin=1in]{geometry}
\usepackage{graphicx}
\usepackage[hidelinks]{hyperref}
\usepackage{xcolor}
\usepackage{fancyhdr}
\usepackage{titlesec}
\usepackage{enumitem}
\usepackage{tcolorbox}
\usepackage{amsmath}
\usepackage{amssymb}
\usepackage{lmodern}
\usepackage[T1]{fontenc}
\usepackage{array}

% Colors
\definecolor{primarygreen}{RGB}{46,125,50}
\definecolor{accentgreen}{RGB}{76,175,80}
\definecolor{lightgray}{RGB}{248,249,250}
\definecolor{bluestat}{RGB}{59,130,246}
\definecolor{anovacolor}{RGB}{34,139,34}
\definecolor{purplestat}{RGB}{147,51,234}

% Header and footer
\setlength{\headheight}{15pt}
\addtolength{\topmargin}{-3pt}
\pagestyle{fancy}
\fancyhf{}
\fancyhead[L]{\textcolor{primarygreen}{\textbf{PLS 120 - Week 10 Tutorial}}}
\fancyhead[R]{\textcolor{primarygreen}{UC Davis}}
\fancyfoot[C]{\thepage}

% Title formatting
\titleformat{\section}{\large\bfseries\color{primarygreen}}{}{0em}{}[\titlerule]
\titleformat{\subsection}{\normalsize\bfseries\color{primarygreen}}{}{0em}{}

% Custom boxes
\newtcolorbox{infobox}{
    colback=lightgray,
    colframe=primarygreen,
    boxrule=1pt,
    arc=3pt,
    left=10pt,
    right=10pt,
    top=10pt,
    bottom=10pt
}

\newtcolorbox{formulabox}{
    colback=bluestat!10,
    colframe=bluestat,
    boxrule=1pt,
    arc=3pt,
    left=10pt,
    right=10pt,
    top=10pt,
    bottom=10pt
}

\newtcolorbox{anovabox}{
    colback=anovacolor!10,
    colframe=anovacolor,
    boxrule=1pt,
    arc=3pt,
    left=10pt,
    right=10pt,
    top=10pt,
    bottom=10pt
}

\newtcolorbox{regressionbox}{
    colback=purplestat!10,
    colframe=purplestat,
    boxrule=1pt,
    arc=3pt,
    left=10pt,
    right=10pt,
    top=10pt,
    bottom=10pt
}

\begin{document}

% Title page
\begin{titlepage}
    \centering
    \vspace*{2cm}
    
    {\Huge\bfseries\color{primarygreen} PLS 120: Applied Statistics in Agricultural Sciences}
    
    \vspace{1cm}
    
    {\Large\color{primarygreen} ANOVA and Linear Regression Analysis}
    
    \vspace{2cm}
    
    \includegraphics[width=0.3\textwidth]{../../images/logos/Home_Page_Logo.png}
    
    \vspace{2cm}
    
    {\large\bfseries Week 10 Tutorial Guide}
    
    \vspace{1cm}
    
    {\large Mohammadreza Narimani}\\
    {\normalsize Department of Biological and Agricultural Engineering}\\
    {\normalsize University of California, Davis}
    
    \vspace{1cm}
    
    {\normalsize mnarimani@ucdavis.edu}
    
    \vfill
    
    {\normalsize December 2025}
\end{titlepage}

\tableofcontents
\newpage

\section{Important Links}

\begin{tcolorbox}[colback=accentgreen!20, colframe=primarygreen, boxrule=2pt, arc=5pt, title={\textbf{\Large Essential Course Resources}}]
\centering
\textbf{\Large Course Website}\\[0.5cm]
\textcolor{primarygreen}{\textbf{All course materials available at:}}\\[0.3cm]
\href{https://mohammadrezanarimaniucdavis.github.io/PLS120-Statistics-Lab-Materials/}{\textcolor{primarygreen}{\underline{Course Website Link}}}\\[0.8cm]

\textbf{\Large Interactive Binder Environment}\\[0.5cm]
\textcolor{primarygreen}{\textbf{Access Week 10 lab materials:}}\\[0.3cm]
\href{https://mybinder.org/v2/gh/MohammadrezaNarimaniUCDavis/PLS120-Statistics-Lab-Materials/binder-week10}{\textcolor{primarygreen}{\underline{Week 10 Binder Link}}}
\end{tcolorbox}

\section{Welcome to Week 10: ANOVA and Linear Regression}

This week, we explore \textbf{Analysis of Variance (ANOVA)} and \textbf{Linear Regression} - powerful statistical methods for comparing multiple groups and modeling relationships between variables in agricultural research.

\section{Analysis of Variance (ANOVA)}

\subsection{When to Use ANOVA}

ANOVA compares means across multiple groups simultaneously, such as testing different diet treatments on pig growth or comparing crop yields across varieties.

\subsection{ANOVA Components and Formulas}

\begin{anovabox}
\textbf{ANOVA Table Structure:}

\begin{center}
\begin{tabular}{|l|c|c|c|c|}
\hline
\textbf{Source} & \textbf{SS} & \textbf{df} & \textbf{MS} & \textbf{F} \\
\hline
Between Groups & SSB & $k-1$ & MSB & $MSB/MSE$ \\
Within Groups & SSE & $N-k$ & MSE & \\
Total & SST & $N-1$ & & \\
\hline
\end{tabular}
\end{center}

\textbf{Sum of Squares Formulas:}

$SSB = \sum_{i=1}^{k} n_i(\bar{x}_i - \bar{x})^2$ (Between groups)

$SSE = \sum_{i=1}^{k} \sum_{j=1}^{n_i} (x_{ij} - \bar{x}_i)^2$ (Within groups)

$SST = \sum_{i=1}^{k} \sum_{j=1}^{n_i} (x_{ij} - \bar{x})^2$ (Total)

\textbf{Mean Squares:}

$MSB = \frac{SSB}{k-1}$ (Between groups mean square)

$MSE = \frac{SSE}{N-k}$ (Error mean square)

\textbf{F-Statistic:}

$F = \frac{MSB}{MSE}$ (Ratio of between to within group variation)
\end{anovabox}

\subsection{ANOVA Hypotheses}

\begin{formulabox}
\textbf{ANOVA Hypotheses:}

\textbf{Null Hypothesis ($H_0$):} All group means are equal

$H_0: \mu_1 = \mu_2 = \mu_3 = \ldots = \mu_k$

\textbf{Alternative Hypothesis ($H_1$):} At least one group mean differs

$H_1:$ Not all $\mu_i$ are equal

\textbf{Decision Rule:}

If $F > F_{critical}$ or $p < \alpha$ $\rightarrow$ Reject $H_0$

If $F \leq F_{critical}$ or $p \geq \alpha$ $\rightarrow$ Fail to reject $H_0$

\textbf{R Implementation:}

\texttt{model <- lm(response \textasciitilde{} factor, data)}

\texttt{anova(model)}

\texttt{summary(model)}
\end{formulabox}

\section{Manual ANOVA Calculations}

\subsection{Step-by-Step ANOVA Construction}

Building ANOVA tables from scratch helps understand the underlying statistical concepts.

\begin{anovabox}
\textbf{Manual ANOVA Steps:}

\textbf{1. Calculate Group Means:}

$\bar{x}_i = \frac{\sum_{j=1}^{n_i} x_{ij}}{n_i}$ for each group $i$

\textbf{2. Calculate Overall Mean:}

$\bar{x} = \frac{\sum_{i=1}^{k} \sum_{j=1}^{n_i} x_{ij}}{N}$ where $N = \sum n_i$

\textbf{3. Calculate Sum of Squares:}

SSB: Weighted squared deviations of group means from overall mean

SST: Squared deviations of all observations from overall mean

SSE: SST - SSB (or sum of within-group squared deviations)

\textbf{4. Calculate Degrees of Freedom:}

Between: $df_1 = k - 1$ (number of groups - 1)

Within: $df_2 = N - k$ (total observations - number of groups)

Total: $df_3 = N - 1$ (total observations - 1)

\textbf{5. Calculate Mean Squares and F-statistic:}

$MSB = SSB / df_1$, $MSE = SSE / df_2$

$F = MSB / MSE$
\end{anovabox}

\section{Linear Regression Analysis}

\subsection{Simple Linear Regression}

Linear regression models the relationship between a continuous predictor and response variable.

\begin{regressionbox}
\textbf{Simple Linear Regression Model:}

$y = \beta_0 + \beta_1 x + \epsilon$

Where:

$y$ = response variable (dependent)

$x$ = predictor variable (independent)

$\beta_0$ = intercept (y-value when x = 0)

$\beta_1$ = slope (change in y per unit change in x)

$\epsilon$ = random error term

\textbf{Least Squares Estimates:}

$\hat{\beta}_1 = \frac{\sum(x_i - \bar{x})(y_i - \bar{y})}{\sum(x_i - \bar{x})^2}$

$\hat{\beta}_0 = \bar{y} - \hat{\beta}_1\bar{x}$

\textbf{Correlation Coefficient:}

$r = \frac{\sum(x_i - \bar{x})(y_i - \bar{y})}{\sqrt{\sum(x_i - \bar{x})^2 \sum(y_i - \bar{y})^2}}$

\textbf{R-squared (Coefficient of Determination):}

$R^2 = r^2$ = Proportion of variance explained by the model

\textbf{R Implementation:}

\texttt{model <- lm(y \textasciitilde{} x, data)}

\texttt{summary(model)}

\texttt{cor(x, y)}
\end{regressionbox}

\subsection{Multiple Linear Regression}

\begin{regressionbox}
\textbf{Multiple Linear Regression Model:}

$y = \beta_0 + \beta_1 x_1 + \beta_2 x_2 + \ldots + \beta_p x_p + \epsilon$

\textbf{Categorical Predictors:}

When using factors (like Species), R creates dummy variables:

Reference category (baseline)

Coefficients represent differences from reference

\textbf{Model Interpretation:}

Each $\beta_i$ represents change in y per unit change in $x_i$

Holding all other variables constant

Intercept is y-value when all predictors = 0

\textbf{R Implementation:}

\texttt{model <- lm(y \textasciitilde{} x1 + x2 + factor, data)}

\texttt{model <- lm(Petal.Width \textasciitilde{} Species, data)}

\texttt{model <- lm(Petal.Width \textasciitilde{} Petal.Length, data)}
\end{regressionbox}

\section{Assignment 10 Overview}

\subsection{Assignment Structure (20 points total)}

\begin{enumerate}
    \item \textbf{Part 1: Data Loading and Study Design (6 points)}
    \begin{itemize}
        \item Identify treatment, response variable, and experimental unit (3 points)
        \item Formulate null and alternative hypotheses (2 points)
        \item Load and examine pig weight data (1 point)
    \end{itemize}
    
    \item \textbf{Part 2: ANOVA Table Calculations (7 points)}
    \begin{itemize}
        \item Calculate sum of squares (SSB, SSE, SST) (2 points)
        \item Calculate degrees of freedom (1 point)
        \item Calculate mean squares and F-statistic (2 points)
        \item Complete ANOVA table (2 points)
    \end{itemize}
    
    \item \textbf{Part 3: Statistical Interpretation and Conclusions (7 points)}
    \begin{itemize}
        \item Interpret F-statistic and p-value (3 points)
        \item Draw conclusions about diet effects (2 points)
        \item Discuss practical implications (2 points)
    \end{itemize}
\end{enumerate}

\section{Agricultural Applications}

\begin{infobox}
\textbf{Real-World ANOVA and Regression Applications:}
\begin{itemize}
    \item \textbf{Variety Trials} - Compare crop performance across multiple varieties
    \item \textbf{Treatment Comparisons} - Test fertilizer, pesticide, or management effects
    \item \textbf{Feed Efficiency Studies} - Analyze animal growth under different diets
    \item \textbf{Soil Management} - Compare tillage or amendment effects on soil health
    \item \textbf{Environmental Factors} - Model relationships between climate and yield
    \item \textbf{Quality Assessment} - Analyze factors affecting product quality
    \item \textbf{Breeding Programs} - Compare genetic lines or breeding methods
    \item \textbf{Economic Analysis} - Model cost-benefit relationships in agriculture
\end{itemize}
\end{infobox}

\section{Getting Started}

\begin{enumerate}
    \item Launch Week 10 Binder environment
    \item Navigate to \texttt{class\_activity} folder
    \item Open \texttt{Week10\_ANOVA\_Regression.ipynb}
    \item Work through ANOVA and regression examples
    \item Complete Assignment 10 in \texttt{assignment} folder
\end{enumerate}

\section{Learning Objectives}

By the end of this week, you will be able to:
\begin{itemize}
    \item Build ANOVA tables from scratch with manual calculations
    \item Understand sum of squares, degrees of freedom, and F-statistics
    \item Perform one-way ANOVA using R functions
    \item Fit and interpret simple and multiple linear regression models
    \item Check assumptions for ANOVA and regression analyses
    \item Choose appropriate statistical methods for different research questions
    \item Interpret statistical output in agricultural research context
    \item Understand the relationship between ANOVA and regression
\end{itemize}

\section{Need Help?}

\begin{infobox}
\textbf{Mohammadreza Narimani}

Email: mnarimani@ucdavis.edu

Department of Biological and Agricultural Engineering, UC Davis

Office Hours: Thursdays 10 AM - 12 PM (Zoom)

Zoom Link: \href{https://ucdavis.zoom.us/j/99533096447}{Join Office Hours}
\end{infobox}

\vfill

\begin{center}
\textit{Last updated: December 2025 | PLS 120 - Applied Statistics in Agriculture | UC Davis}

\textit{Week 10: ANOVA and Linear Regression Analysis}
\end{center}

\end{document}