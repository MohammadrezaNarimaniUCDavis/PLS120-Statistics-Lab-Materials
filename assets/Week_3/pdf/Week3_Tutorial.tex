\documentclass[11pt,a4paper]{article}
\usepackage[utf8]{inputenc}
\usepackage[margin=1in]{geometry}
\usepackage{graphicx}
\usepackage[hidelinks]{hyperref}
\usepackage{xcolor}
\usepackage{fancyhdr}
\usepackage{titlesec}
\usepackage{enumitem}
\usepackage{tcolorbox}
\usepackage{fontawesome5}
\usepackage{amsmath}
\usepackage{amssymb}
\usepackage{lmodern}
\usepackage[T1]{fontenc}
\usepackage{needspace}

% Colors
\definecolor{primarygreen}{RGB}{46,125,50}
\definecolor{accentgreen}{RGB}{76,175,80}
\definecolor{lightgray}{RGB}{248,249,250}
\definecolor{bluestat}{RGB}{59,130,246}
\definecolor{redstat}{RGB}{239,68,68}

% Header and footer
\setlength{\headheight}{15pt}
\addtolength{\topmargin}{-3pt}
\pagestyle{fancy}
\fancyhf{}
\fancyhead[L]{\textcolor{primarygreen}{\textbf{PLS 120 - Week 3 Tutorial}}}
\fancyhead[R]{\textcolor{primarygreen}{UC Davis}}
\fancyfoot[C]{\thepage}

% Title formatting
\titleformat{\section}{\Large\bfseries\color{primarygreen}}{}{0em}{}[\titlerule]
\titleformat{\subsection}{\large\bfseries\color{primarygreen}}{}{0em}{}

% Custom boxes
\newtcolorbox{infobox}{
    colback=lightgray,
    colframe=primarygreen,
    boxrule=1pt,
    arc=3pt,
    left=10pt,
    right=10pt,
    top=10pt,
    bottom=10pt
}

\newtcolorbox{warningbox}{
    colback=accentgreen!10,
    colframe=primarygreen,
    boxrule=1pt,
    arc=3pt,
    left=10pt,
    right=10pt,
    top=10pt,
    bottom=10pt
}

\newtcolorbox{formulabox}{
    colback=bluestat!10,
    colframe=bluestat,
    boxrule=1pt,
    arc=3pt,
    left=10pt,
    right=10pt,
    top=10pt,
    bottom=10pt
}

\begin{document}

% Title page
\begin{titlepage}
    \centering
    \vspace*{2cm}
    
    {\Huge\bfseries\color{primarygreen} PLS 120: Applied Statistics in Agricultural Sciences}
    
    \vspace{1cm}
    
    {\Large\color{primarygreen} Data Manipulation with dplyr}
    
    \vspace{2cm}
    
    \includegraphics[width=0.3\textwidth]{../../images/logos/Home_Page_Logo.png}
    
    \vspace{2cm}
    
    {\large\bfseries Week 3 Tutorial Guide}
    
    \vspace{1cm}
    
    {\large Mohammadreza Narimani}\\
    {\normalsize Department of Biological and Agricultural Engineering}\\
    {\normalsize University of California, Davis}
    
    \vspace{1cm}
    
    {\normalsize mnarimani@ucdavis.edu}
    
    \vfill
    
    {\normalsize October 2025}
\end{titlepage}

\tableofcontents
\newpage

\section{Important Links}

\begin{tcolorbox}[colback=accentgreen!20, colframe=primarygreen, boxrule=2pt, arc=5pt, title={\textbf{\Large Essential Course Resources}}]
\centering
\textbf{\Large Course Website}\\[0.5cm]
\textcolor{primarygreen}{\textbf{All course materials available at:}}\\[0.3cm]
\href{https://mohammadrezanarimaniucdavis.github.io/PLS120-Statistics-Lab-Materials/}{\textcolor{primarygreen}{\underline{Course Website Link}}}\\[0.8cm]

\textbf{\Large Interactive Binder Environment}\\[0.5cm]
\textcolor{primarygreen}{\textbf{Access Week 3 lab materials:}}\\[0.3cm]
\href{https://mybinder.org/v2/gh/MohammadrezaNarimaniUCDavis/PLS120-Statistics-Lab-Materials/binder-week3}{\textcolor{primarygreen}{\underline{Week 3 Binder Link}}}
\end{tcolorbox}

\section{Welcome to Week 3: Data Manipulation with dplyr}

This week, we explore \textbf{data manipulation using dplyr} and the \textbf{tidyverse ecosystem} - essential tools for organizing, cleaning, and visualizing agricultural data. You'll learn to filter, select, arrange, and transform data efficiently!

\section{Key Data Manipulation Concepts}

\subsection{Basic Data Subsetting}

Understanding how to extract specific parts of your data is fundamental to data analysis.

\subsubsection{Bracket Notation}

\begin{formulabox}
\textbf{Syntax:} \texttt{data[row, column]}\\[0.3cm]
\textbf{Examples:}\\
\texttt{data[1,1]} - Single cell\\
\texttt{data[1:5, ]} - First 5 rows\\
\texttt{data[, 1:3]} - First 3 columns\\
\texttt{data[1:10, 2:4]} - Specific rows and columns
\end{formulabox}

\subsection{The Power of Pipes (\%>\%)}

\begin{formulabox}
\textbf{Concept:} Chain operations together for readable code\\[0.3cm]
\textbf{Think of pipes as "then":} Take data, \textbf{then} filter, \textbf{then} select\\[0.3cm]
\textbf{Example:} \texttt{data \%>\% filter(Species == "setosa") \%>\% select(Sepal.Length)}
\end{formulabox}

\needspace{5\baselineskip}
\section{Essential dplyr Functions}

\subsection{Filtering Data}

\subsubsection{filter() Function}

\begin{formulabox}
\textbf{Purpose:} Subset rows based on conditions\\[0.3cm]
\textbf{Syntax:} \texttt{filter(data, condition)}\\[0.3cm]
\textbf{Examples:}\\
\texttt{filter(Species == "setosa")} - Exact match\\
\texttt{filter(Sepal.Length > 5)} - Numerical condition\\
\texttt{filter(Species \%in\% c("setosa", "virginica"))} - Multiple values
\end{formulabox}

\subsubsection{slice() Functions}

\begin{formulabox}
\textbf{Purpose:} Select rows by position\\[0.3cm]
\textbf{Functions:}\\
\texttt{slice(10:20)} - Rows 10 to 20\\
\texttt{slice\_head(n=5)} - First 5 rows\\
\texttt{slice\_tail(n=5)} - Last 5 rows\\
\texttt{slice\_sample(n=10)} - Random 10 rows
\end{formulabox}

\subsection{Selecting Columns}

\subsubsection{select() Function}

\begin{formulabox}
\textbf{Purpose:} Choose specific columns\\[0.3cm]
\textbf{Basic Selection:}\\
\texttt{select(Sepal.Length, Species)} - By name\\
\texttt{select(1, 5)} - By position\\
\texttt{select(1:3)} - Range of columns
\end{formulabox}

\subsubsection{Advanced Selection Helpers}

\begin{formulabox}
\textbf{Pattern Matching:}\\
\texttt{starts\_with("Sepal")} - Columns starting with "Sepal"\\
\texttt{ends\_with("Length")} - Columns ending with "Length"\\
\texttt{contains("Petal")} - Columns containing "Petal"\\
\texttt{matches(".*Width")} - Regular expression matching
\end{formulabox}

\subsection{Data Transformation}

\subsubsection{arrange() Function}

\begin{formulabox}
\textbf{Purpose:} Sort data by variables\\[0.3cm]
\textbf{Examples:}\\
\texttt{arrange(Sepal.Length)} - Ascending order\\
\texttt{arrange(desc(Sepal.Length))} - Descending order\\
\texttt{arrange(Species, Sepal.Length)} - Multiple variables
\end{formulabox}

\subsubsection{mutate() Function}

\begin{formulabox}
\textbf{Purpose:} Create new variables or transform existing ones\\[0.3cm]
\textbf{Examples:}\\
\texttt{mutate(Sepal.Area = Sepal.Length * Sepal.Width)}\\
\texttt{mutate(Length\_mm = Sepal.Length * 10)}\\
\texttt{mutate(Size\_Category = ifelse(Sepal.Length > 5, "Large", "Small"))}
\end{formulabox}

\subsubsection{rename() Function}

\begin{formulabox}
\textbf{Purpose:} Change column names\\[0.3cm]
\textbf{Syntax:} \texttt{rename(new\_name = old\_name)}\\[0.3cm]
\textbf{Example:} \texttt{rename(Sepal\_Length = Sepal.Length, Plant\_Species = Species)}
\end{formulabox}

\subsection{Grouping and Summarizing}

\subsubsection{group\_by() and summarize()}

\begin{formulabox}
\textbf{Purpose:} Calculate statistics for different groups\\[0.3cm]
\textbf{Workflow:}\\
1. Group data by categorical variable\\
2. Calculate summary statistics for each group\\[0.3cm]
\textbf{Example:}\\
\texttt{data \%>\% group\_by(Species) \%>\%}\\[0.1cm]
\texttt{summarize(mean\_length = mean(Sepal.Length))}
\end{formulabox}

\needspace{5\baselineskip}
\section{Data Visualization}

\subsection{ggplot2 Fundamentals}

\subsubsection{Basic ggplot Structure}

\begin{formulabox}
\textbf{Components:}\\
1. \texttt{ggplot(data, aes(x = variable, y = variable))} - Base layer\\
2. \texttt{+ geom\_*()} - Add geometric objects\\
3. \texttt{+ labs()} - Add labels and titles\\[0.3cm]
\textbf{Example:}\\
\texttt{ggplot(data, aes(x = Species, y = Sepal.Length)) + geom\_boxplot()}
\end{formulabox}

\subsubsection{Common Plot Types}

\begin{formulabox}
\textbf{Histograms:} \texttt{geom\_histogram()} - Distribution of single variable\\
\textbf{Boxplots:} \texttt{geom\_boxplot()} - Compare groups\\
\textbf{Scatter plots:} \texttt{geom\_point()} - Relationship between variables\\
\textbf{Bar plots:} \texttt{geom\_bar()} - Count categorical data\\
\textbf{Density plots:} \texttt{geom\_density()} - Smooth distribution curves
\end{formulabox}

\subsection{Base R Plotting}

\subsubsection{Quick Visualization Functions}

\begin{formulabox}
\textbf{Histograms:} \texttt{hist(data\$variable)}\\
\textbf{Boxplots:} \texttt{boxplot(variable \textasciitilde{} group, data = data)}\\
\textbf{Scatter plots:} \texttt{plot(x, y)}\\
\textbf{Stem-and-leaf:} \texttt{stem(data\$variable)}
\end{formulabox}

\needspace{5\baselineskip}
\section{Data Cleaning Techniques}

\subsection{Handling Non-Numeric Data}

\subsubsection{String Manipulation}

\begin{formulabox}
\textbf{Remove non-numeric characters:}\\
\texttt{str\_replace\_all(text, "[\textasciicircum{}0-9]", "")}\\[0.3cm]
\textbf{Convert to numeric:}\\
\texttt{as.integer(character\_vector)}\\
\texttt{as.numeric(character\_vector)}
\end{formulabox}

\subsubsection{Missing Value Treatment}

\begin{formulabox}
\textbf{Remove missing values:} \texttt{na.omit(data)}\\
\textbf{Check for missing values:} \texttt{is.na(data)}\\
\textbf{Count missing values:} \texttt{sum(is.na(data))}
\end{formulabox}

\needspace{5\baselineskip}
\section{Assignment 3 Overview}

\subsection{Assignment Structure (20 points total)}

\begin{enumerate}
    \item \textbf{Part 1: LA Data Analysis (6 points)}
    \begin{itemize}
        \item Load and filter data by gender
        \item Create comparative boxplots
        \item Interpret statistical differences
    \end{itemize}
    
    \item \textbf{Part 2: SAT Dataset Processing (9 points)}
    \begin{itemize}
        \item Import and inspect data
        \item Create random subsets
        \item Clean non-numeric values
        \item Extract specific columns
    \end{itemize}
    
    \item \textbf{Part 3: Distribution Analysis (5 points)}
    \begin{itemize}
        \item Create stem-and-leaf plots
        \item Analyze distribution patterns
        \item Identify outliers and central tendency
    \end{itemize}
\end{enumerate}

\needspace{5\baselineskip}
\section{Agricultural Applications}

\begin{infobox}
\textbf{Real-World Applications:}
\begin{itemize}
    \item \textbf{Crop Yield Analysis} - Filter by variety, location, season
    \item \textbf{Soil Sample Processing} - Clean mixed numeric/text data
    \item \textbf{Weather Pattern Analysis} - Group by month, calculate averages
    \item \textbf{Livestock Performance} - Compare treatments, identify outliers
    \item \textbf{Quality Control} - Monitor product consistency over time
    \item \textbf{Field Trial Analysis} - Subset by treatment groups
\end{itemize}
\end{infobox}

\section{Getting Started}

\begin{enumerate}
    \item Launch Week 3 Binder environment
    \item Navigate to \texttt{class\_activity} folder
    \item Open \texttt{Week3\_Data\_Manipulation.ipynb}
    \item Work through interactive exercises
\end{enumerate}

\section{Learning Objectives}

By the end of this week, you will be able to:
\begin{itemize}
    \item Master data subsetting with brackets and logical conditions
    \item Apply dplyr functions for efficient data manipulation
    \item Chain operations using pipes for readable workflows
    \item Select columns using helper functions
    \item Clean real-world data with mixed data types
    \item Create visualizations to understand data patterns
    \item Interpret statistical distributions and identify outliers
\end{itemize}

\section{Need Help?}

\begin{infobox}
\textbf{Mohammadreza Narimani}\\
Email: mnarimani@ucdavis.edu\\
Department of Biological and Agricultural Engineering, UC Davis\\
Office Hours: Thursdays 10 AM - 12 PM (Zoom)
\end{infobox}

\vfill

\begin{center}
\textit{Last updated: October 2025 | PLS 120 - Applied Statistics in Agriculture | UC Davis}\\
\textit{Week 3: Data Manipulation with dplyr}
\end{center}

\end{document}