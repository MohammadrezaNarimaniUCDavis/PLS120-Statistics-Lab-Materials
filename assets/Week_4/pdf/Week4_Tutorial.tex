\documentclass[11pt,a4paper]{article}
\usepackage[utf8]{inputenc}
\usepackage[margin=1in]{geometry}
\usepackage{graphicx}
\usepackage[hidelinks]{hyperref}
\usepackage{xcolor}
\usepackage{fancyhdr}
\usepackage{titlesec}
\usepackage{enumitem}
\usepackage{tcolorbox}
\usepackage{fontawesome5}
\usepackage{amsmath}
\usepackage{amssymb}
\usepackage{lmodern}
\usepackage[T1]{fontenc}
\usepackage{needspace}

% Colors
\definecolor{primarygreen}{RGB}{46,125,50}
\definecolor{accentgreen}{RGB}{76,175,80}
\definecolor{lightgray}{RGB}{248,249,250}
\definecolor{bluestat}{RGB}{59,130,246}
\definecolor{redstat}{RGB}{239,68,68}
\definecolor{purplestat}{RGB}{147,51,234}

% Header and footer
\setlength{\headheight}{15pt}
\addtolength{\topmargin}{-3pt}
\pagestyle{fancy}
\fancyhf{}
\fancyhead[L]{\textcolor{primarygreen}{\textbf{PLS 120 - Week 4 Tutorial}}}
\fancyhead[R]{\textcolor{primarygreen}{UC Davis}}
\fancyfoot[C]{\thepage}

% Title formatting
\titleformat{\section}{\Large\bfseries\color{primarygreen}}{}{0em}{}[\titlerule]
\titleformat{\subsection}{\large\bfseries\color{primarygreen}}{}{0em}{}

% Custom boxes
\newtcolorbox{infobox}{
    colback=lightgray,
    colframe=primarygreen,
    boxrule=1pt,
    arc=3pt,
    left=10pt,
    right=10pt,
    top=10pt,
    bottom=10pt
}

\newtcolorbox{warningbox}{
    colback=accentgreen!10,
    colframe=primarygreen,
    boxrule=1pt,
    arc=3pt,
    left=10pt,
    right=10pt,
    top=10pt,
    bottom=10pt
}

\newtcolorbox{formulabox}{
    colback=bluestat!10,
    colframe=bluestat,
    boxrule=1pt,
    arc=3pt,
    left=10pt,
    right=10pt,
    top=10pt,
    bottom=10pt
}

\newtcolorbox{probabilitybox}{
    colback=purplestat!10,
    colframe=purplestat,
    boxrule=1pt,
    arc=3pt,
    left=10pt,
    right=10pt,
    top=10pt,
    bottom=10pt
}

\begin{document}

% Title page
\begin{titlepage}
    \centering
    \vspace*{2cm}
    
    {\Huge\bfseries\color{primarygreen} PLS 120: Applied Statistics in Agricultural Sciences}
    
    \vspace{1cm}
    
    {\Large\color{primarygreen} Probability and Sampling}
    
    \vspace{2cm}
    
    \includegraphics[width=0.3\textwidth]{../../images/logos/Home_Page_Logo.png}
    
    \vspace{2cm}
    
    {\large\bfseries Week 4 Tutorial Guide}
    
    \vspace{1cm}
    
    {\large Mohammadreza Narimani}\\
    {\normalsize Department of Biological and Agricultural Engineering}\\
    {\normalsize University of California, Davis}
    
    \vspace{1cm}
    
    {\normalsize mnarimani@ucdavis.edu}
    
    \vfill
    
    {\normalsize October 2025}
\end{titlepage}

\tableofcontents
\newpage

\section{Important Links}

\begin{tcolorbox}[colback=accentgreen!20, colframe=primarygreen, boxrule=2pt, arc=5pt, title={\textbf{\Large Essential Course Resources}}]
\centering
\textbf{\Large Course Website}\\[0.5cm]
\textcolor{primarygreen}{\textbf{All course materials available at:}}\\[0.3cm]
\href{https://mohammadrezanarimaniucdavis.github.io/PLS120-Statistics-Lab-Materials/}{\textcolor{primarygreen}{\underline{Course Website Link}}}\\[0.8cm]

\textbf{\Large Interactive Binder Environment}\\[0.5cm]
\textcolor{primarygreen}{\textbf{Access Week 4 lab materials:}}\\[0.3cm]
\href{https://mybinder.org/v2/gh/MohammadrezaNarimaniUCDavis/PLS120-Statistics-Lab-Materials/binder-week4}{\textcolor{primarygreen}{\underline{Week 4 Binder Link}}}
\end{tcolorbox}

\section{Welcome to Week 4: Probability and Sampling}

This week, we explore \textbf{probability theory and sampling techniques} - essential foundations for statistical inference in agricultural research. You'll learn to simulate probability experiments, work with distributions, and understand randomness!

\section{Logical Variables and Data Types}

\subsection{Understanding Logical Variables}

Logical variables in R can only take the values TRUE, FALSE, or NA (missing value). They are fundamental in decision-making processes in programming.

\subsubsection{Basic Logical Operations}

\begin{formulabox}
\textbf{Comparison Operators:}\\
\texttt{==} - Equal to\\
\texttt{!=} - Not equal to\\
\texttt{<} - Less than\\
\texttt{>} - Greater than\\
\texttt{<=} - Less than or equal to\\
\texttt{>=} - Greater than or equal to\\[0.3cm]
\textbf{Logical Operators:}\\
\texttt{\&} - AND (element-wise)\\
\texttt{|} - OR (element-wise)\\
\texttt{!} - NOT (negation)
\end{formulabox}

\subsection{Data Type Conversion}

\subsubsection{Essential Conversion Functions}

\begin{formulabox}
\textbf{Type Conversion Functions:}\\
\texttt{as.numeric(x)} - Convert to numeric\\
\texttt{as.character(x)} - Convert to character\\
\texttt{as.factor(x)} - Convert to factor\\
\texttt{as.logical(x)} - Convert to logical\\
\texttt{data.frame(x)} - Convert to data frame\\[0.3cm]
\textbf{Example:}\\
\texttt{numeric\_vector <- c(0, 1, 2)}\\
\texttt{logical\_vector <- as.logical(numeric\_vector)}\\
\texttt{\# Result: FALSE, TRUE, TRUE}
\end{formulabox}

\needspace{5\baselineskip}
\section{Random Sampling Techniques}

\subsection{The sample() Function}

Understanding how to perform random sampling from a population is essential for statistical analysis.

\subsubsection{Function Parameters}

\begin{formulabox}
\textbf{Syntax:} \texttt{sample(x, size, replace)}\\[0.3cm]
\textbf{Parameters:}\\
\texttt{x} - Population (dataset) to sample from\\
\texttt{size} - Number of samples to draw\\
\texttt{replace} - TRUE (with replacement) or FALSE (without replacement)\\[0.3cm]
\textbf{Examples:}\\
\texttt{sample(1:6, 10, replace = TRUE)} - Roll die 10 times\\
\texttt{sample(nrow(data), 30, replace = FALSE)} - Select 30 unique rows
\end{formulabox}

\subsubsection{Reproducible Results}

\begin{formulabox}
\textbf{set.seed() Function:}\\
Use \texttt{set.seed(number)} before random sampling to ensure reproducible results\\[0.3cm]
\textbf{Example:}\\
\texttt{set.seed(123)}\\
\texttt{sample(c("H", "T"), 10, replace = TRUE)}\\
\texttt{\# Will always produce the same sequence}
\end{formulabox}

\needspace{5\baselineskip}
\section{Probability Simulation}

\subsection{Coin Toss Experiments}

\subsubsection{Basic Coin Simulation}

\begin{probabilitybox}
\textbf{Create a Coin:} \texttt{coin <- c("H", "T")}\\[0.3cm]
\textbf{Simulate Tosses:} \texttt{tosses <- sample(coin, size = 50, replace = TRUE)}\\[0.3cm]
\textbf{Count Outcomes:}\\
\texttt{heads <- sum(tosses == "H")}\\
\texttt{tails <- sum(tosses == "T")}\\[0.3cm]
\textbf{Calculate Probabilities:}\\
\texttt{prob\_heads <- heads / length(tosses)}\\
\texttt{prob\_tails <- tails / length(tosses)}
\end{probabilitybox}

\subsubsection{Frequency Analysis}

\begin{probabilitybox}
\textbf{Create Frequency Table:} \texttt{toss\_table <- table(tosses)}\\[0.3cm]
\textbf{Convert to Probabilities:}\\
\texttt{toss\_probabilities <- toss\_table / sum(toss\_table)}\\[0.3cm]
\textbf{Theoretical vs Experimental:}\\
Theoretical probability for fair coin: P(H) = P(T) = 0.5\\
Experimental probability varies with sample size
\end{probabilitybox}

\subsection{Dice Roll Simulation}

\subsubsection{Single Die Experiments}

\begin{probabilitybox}
\textbf{Create a Die:} \texttt{dice <- c(1:6)} or \texttt{dice <- seq(1, 6, 1)}\\[0.3cm]
\textbf{Simulate Rolls:} \texttt{rolls <- sample(dice, size = 100, replace = TRUE)}\\[0.3cm]
\textbf{Analyze Results:}\\
\texttt{roll\_counts <- table(rolls)}\\
\texttt{roll\_probabilities <- roll\_counts / sum(roll\_counts)}\\[0.3cm]
\textbf{Theoretical:} Each face should have probability = 1/6 $\approx$ 0.167
\end{probabilitybox}

\subsubsection{Two Dice and Central Limit Theorem}

\begin{probabilitybox}
\textbf{Sum of Two Dice:}\\
\texttt{die1 <- sample(1:6, 1000, replace = TRUE)}\\
\texttt{die2 <- sample(1:6, 1000, replace = TRUE)}\\
\texttt{sums <- die1 + die2}\\[0.3cm]
\textbf{Observe Distribution:}\\
As sample size increases, the distribution of sums approaches normal distribution (Central Limit Theorem)\\[0.3cm]
\textbf{Theoretical Mean:} E(sum) = 7\\
\textbf{Theoretical SD:} $\sigma$(sum) $\approx$ 2.42
\end{probabilitybox}

\needspace{5\baselineskip}
\section{Normal Distribution Functions}

\subsection{Generating Random Normal Data}

\subsubsection{rnorm() Function}

\begin{formulabox}
\textbf{Purpose:} Generate random numbers from normal distribution\\[0.3cm]
\textbf{Syntax:} \texttt{rnorm(n, mean = 0, sd = 1)}\\[0.3cm]
\textbf{Parameters:}\\
\texttt{n} - Number of random values to generate\\
\texttt{mean} - Mean of the distribution (default: 0)\\
\texttt{sd} - Standard deviation (default: 1)\\[0.3cm]
\textbf{Example:} \texttt{normal\_data <- rnorm(100, mean = 50, sd = 15)}
\end{formulabox}

\subsection{Probability Calculations}

\subsubsection{pnorm() Function}

\begin{formulabox}
\textbf{Purpose:} Calculate cumulative probability (area under curve)\\[0.3cm]
\textbf{Syntax:} \texttt{pnorm(q, mean = 0, sd = 1)}\\[0.3cm]
\textbf{Returns:} P(X $\leq$ q) for normal distribution\\[0.3cm]
\textbf{Example:}\\
\texttt{prob\_less\_than\_60 <- pnorm(60, mean = 50, sd = 10)}\\
\texttt{\# Returns probability that X < 60}
\end{formulabox}

\subsubsection{qnorm() Function}

\begin{formulabox}
\textbf{Purpose:} Find quantiles (inverse of pnorm)\\[0.3cm]
\textbf{Syntax:} \texttt{qnorm(p, mean = 0, sd = 1)}\\[0.3cm]
\textbf{Returns:} Value x such that P(X $\leq$ x) = p\\[0.3cm]
\textbf{Example:}\\
\texttt{value\_at\_90th <- qnorm(0.90, mean = 50, sd = 10)}\\
\texttt{\# Returns value below which 90\% of data falls}
\end{formulabox}

\subsection{Visual Probability Functions}

\subsubsection{tigerstats Package}

\begin{formulabox}
\textbf{Enhanced Visualization:}\\
\texttt{library(tigerstats)}\\[0.3cm]
\texttt{pnormGC(60, mean = 50, sd = 10, graph = TRUE)}\\
\texttt{\# Shows probability with visual graph}\\[0.3cm]
\texttt{qnormGC(0.90, mean = 50, sd = 10, graph = TRUE)}\\
\texttt{\# Shows quantile with visual graph}
\end{formulabox}

\needspace{5\baselineskip}
\section{Data Visualization for Probability}

\subsection{Creating Probability Plots}

\subsubsection{Bar Plots for Discrete Distributions}

\begin{formulabox}
\textbf{Base R Approach:}\\
\texttt{barplot(probability\_table, main = "Probability Distribution")}\\[0.3cm]
\textbf{ggplot2 Approach:}\\
\texttt{ggplot(data\_frame, aes(x = outcome, y = probability)) +}\\
\texttt{geom\_bar(stat = "identity")}\\[0.3cm]
\textbf{Note:} Use \texttt{stat = "identity"} to plot actual probability values
\end{formulabox}

\subsubsection{Histograms for Continuous Distributions}

\begin{formulabox}
\textbf{For Normal Data:}\\
\texttt{hist(normal\_data, breaks = 15)}\\[0.3cm]
\textbf{With ggplot2:}\\
\texttt{ggplot(data.frame(x = normal\_data), aes(x = x)) +}\\
\texttt{geom\_histogram(bins = 15)}\\[0.3cm]
\textbf{Density Plots:}\\
\texttt{ggplot(data.frame(x = normal\_data), aes(x = x)) +}\\
\texttt{geom\_density()}
\end{formulabox}

\needspace{5\baselineskip}
\section{Assignment 4 Overview}

\subsection{Assignment Structure (20 points total)}

\begin{enumerate}
    \item \textbf{Part 1: Simulation (6 points)}
    \begin{itemize}
        \item Simulate 50 coin flips (3 points)
        \item Simulate 50 dice rolls (3 points)
    \end{itemize}
    
    \item \textbf{Part 2: Probability Calculation (6 points)}
    \begin{itemize}
        \item Calculate experimental probabilities for coin outcomes (2 points)
        \item Calculate experimental probabilities for dice outcomes (3 points)
        \item Compare experimental vs theoretical probabilities (1 point)
    \end{itemize}
    
    \item \textbf{Part 3: Data Frames and Visualization (8 points)}
    \begin{itemize}
        \item Create coin probability data frame (2 points)
        \item Create dice probability data frame (2 points)
        \item Generate coin flip bar plot (2 points)
        \item Generate dice roll bar plot (2 points)
    \end{itemize}
\end{enumerate}

\needspace{5\baselineskip}
\section{Agricultural Applications}

\begin{infobox}
\textbf{Real-World Applications:}
\begin{itemize}
    \item \textbf{Seed Germination Studies} - Model probability of germination success under different conditions
    \item \textbf{Weather Risk Assessment} - Simulate probability of drought, frost, or extreme weather events
    \item \textbf{Quality Control Sampling} - Random sampling of agricultural products for testing
    \item \textbf{Field Trial Design} - Understanding sampling variability in experimental plots
    \item \textbf{Pest Management} - Modeling probability distributions of pest occurrence
    \item \textbf{Crop Insurance} - Calculating risk probabilities for insurance premium determination
\end{itemize}
\end{infobox}

\section{Key Concepts Summary}

\subsection{Probability Fundamentals}

\begin{probabilitybox}
\textbf{Basic Probability Rules:}\\
• Probability ranges from 0 to 1\\
• P(Event) = Favorable outcomes / Total outcomes\\
• Sum of all probabilities = 1\\
• P(not A) = 1 - P(A)\\[0.3cm]
\textbf{Law of Large Numbers:}\\
As sample size increases, experimental probability approaches theoretical probability
\end{probabilitybox}

\subsection{Sampling Concepts}

\begin{probabilitybox}
\textbf{Sampling with Replacement:}\\
Each item can be selected multiple times (like rolling dice)\\[0.3cm]
\textbf{Sampling without Replacement:}\\
Each item can only be selected once (like drawing cards without putting back)\\[0.3cm]
\textbf{Population vs Sample:}\\
Population = entire group; Sample = subset of population
\end{probabilitybox}

\section{Getting Started}

\begin{enumerate}
    \item Launch Week 4 Binder environment
    \item Navigate to \texttt{class\_activity} folder
    \item Open \texttt{Week4\_Probability\_Sampling.ipynb}
    \item Work through interactive exercises
    \item Complete Assignment 4 in \texttt{assignment} folder
\end{enumerate}

\section{Learning Objectives}

By the end of this week, you will be able to:
\begin{itemize}
    \item Understand logical variables and data type conversions
    \item Perform random sampling with and without replacement
    \item Simulate probability experiments (coins, dice)
    \item Work with normal distribution functions (rnorm, pnorm, qnorm)
    \item Compare experimental and theoretical probabilities
    \item Visualize probability distributions with bar plots and histograms
    \item Apply probability concepts to agricultural research scenarios
\end{itemize}

\section{Tips for Success}

\begin{warningbox}
\textbf{Best Practices:}
\begin{itemize}
    \item Use \texttt{set.seed()} for reproducible random results
    \item Start with small sample sizes to understand concepts
    \item Always verify that probabilities sum to 1
    \item Compare experimental results to theoretical expectations
    \item Use visualization to understand probability distributions
\end{itemize}
\end{warningbox}

\section{Need Help?}

\begin{infobox}
\textbf{Mohammadreza Narimani}\\
Email: mnarimani@ucdavis.edu\\
Department of Biological and Agricultural Engineering, UC Davis\\
Office Hours: Thursdays 10 AM - 12 PM (Zoom)
\end{infobox}

\vfill

\begin{center}
\textit{Last updated: October 2025 | PLS 120 - Applied Statistics in Agriculture | UC Davis}\\
\textit{Week 4: Probability and Sampling}
\end{center}

\end{document}